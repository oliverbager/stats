\subsection{Chapter problems}
\subsubsection{Chapter 1}
\paragraph{Problem 1}
Suppose the universal set is defined as $S=\{x\in \mathbb{N} | 1\geq x \geq 10\}$, $A=\{1,2,3\}$, $B=\{x\in S|2\geq x\geq 7\}$ and $C=\{7,8,9,10\}$.

a) Find $A\cup B$
\[
    A\cup B=\{1,2,3\}\cup\{2,3,4,5,6,7\}=\{1,2,3,4,5,6,7\}
\]
b) Find $(A\cup C)\setminus B$
\[
    (A\cup C)\setminus B=\{1,2,3,7,8,9,10\}\setminus \{2,3,4,5,6,7\}=\{1,8,9,10\}
\]
c) Find $\overline{A}\cup(B\setminus C)$
\[
    \overline{A}\cup(B\setminus C)=\{4,5,6,7,8,9,10\}\cup\{2,3,4,5,6\}=\{2,3,4,5,6,7,8,9,10\}
\]
d) Do $A,B$ and $C$ form a partition of $S$?

No as $A\cap B\neq\emptyset$ and $B\cap C\neq\emptyset$.
\paragraph{Problem 2}
When working with real numbers, our universal set is $\mathbb{R}$. Find each of the following sets
a) $[6,8]\cup[2,7[$
\[
    [2,8]
\]
b) $[6,8]\cap[2,7[$
\[
    [6,7[
\]
c) $\overline{[0,1]}$
\[
    ]-\infty,0[\cup]1,\infty[
\]
d) $[6,8]\setminus ]2,7[$
\[
    [7,8]
\]
\paragraph{Problem 3}
For each of the follopwing Venn diagrams, write the sets denoted by the shaded area

a)
\[
    A\cup B\setminus A\cap B
\]
b)
\[
    B\setminus C
\]
c)
\[
    A\cap B+A\cap C
\]
d)
\[
    C\cup(A\cap B)\setminus ((C\cap A)\cup (C\cap B))
\]
\paragraph{Problem 4}
A coin is tossed twice. Let $S=\{H,T\}\times\{H,T\}$. Write the following sets by listing their elements:

a) First toss is heads
\[
    A=\{(H,T),(H,H)\}
\]
b) At least one tails
\[
    B=\{(T,T),(T,H),(H,T)\}
\]
c) Two tosses are different
\[
    C=\{(H,T),(T,H)\}
\]
\paragraph{Problem 5}
Let $A=\{1,2,\ldots,100\}$. For any $i\in \mathbb{N}$, define $A_{i}$ as the set of numbers in $A$ that are divisible by $i$.

a)
\begin{align*}
    |A_{1}|&=\left\lfloor{\frac{|S|}{1}}\right\rfloor=\left\lfloor{\frac{100}{1}}\right\rfloor=100 \\
    |A_{2}|&=\left\lfloor{\frac{|S|}{2}}\right\rfloor=\left\lfloor{\frac{100}{2}}\right\rfloor=50 \\
    |A_{3}|&=\left\lfloor{\frac{|S|}{3}}\right\rfloor=\left\lfloor{\frac{100}{3}}\right\rfloor=33 \\
    |A_{4}|&=\left\lfloor{\frac{|S|}{4}}\right\rfloor=\left\lfloor{\frac{100}{4}}\right\rfloor=25 \\
    |A_{5}|&=\left\lfloor{\frac{|S|}{5}}\right\rfloor=\left\lfloor{\frac{100}{5}}\right\rfloor=20
\end{align*}
b)

By the distributive property we have that
\[
    A_2\cap A_3\cap A_5=(A_2\cap A_3)\cap A_5
\]
As the intersection of $A_2$ and $A_3$ must be the even factors of 3 we get
\[
    A_2\cap A_3=\{x\in A|x=6n,n\in \mathbb{N}\}
\]
The cardinality of this must be half of the original as every other value is valid
\[
    |A_2\cap A_3|=\left\lfloor{\frac{33}{2}}\right\rfloor=16
\]
As only every fifth factor of 6 is a factor of 5 we get that
\[
    A_2\cap A_3\cap A_5=\{x\in A|x=30n,n\in \mathbb{N}\}
\]
As this is every fifth value of the previous intersection, the cardinality must be one fifth of the previous
\[
    |A_2\cap A_3\cap A_5|=\left\lfloor{\frac{16}{5}}\right\rfloor=3
\]

\paragraph{Problem 6}
As $A_1,A_2,A_3$ form a partition of the universal set, the cardinality of $B$ must be equal to the sum of the cardinalities in the individual partitions.
\[
    |B|=\sum_{i=1}^{3} |B\cap A_i|=10+20+15=45
\]
\paragraph{Problem 7}
a)
As the numbers can be listed in one-to-one correspondance with the natural numbers the set is countable.

b) As the set is made up of the union of 2 countable sets we have that
\[
    B=\bigcup_{i=\mathbb{Q}}\bigcup_{j=\mathbb{Q}}\{a_i+b_j\sqrt{2}\}
\]
As such it must be countable as its constituents are.

c) As the set is a subset of an uncountable set (real numbers), it is not countable.

\paragraph{Problem 8}
We take the limit of the upper bound of the interval as $n\rightarrow\infty$
\[
    \lim_{n\rightarrow\infty}\frac{n-1}{n}=1
\]
As this limit is the upper bound of the interval, corresponding to the maximal interval included in the union of sets we get that
\[
    A=\bigcup_{n=1}^{\infty}A_n=[0;1[
\]
\paragraph{Problem 9}
Opposed to the previous problem the smallest set will here define the set as the intersection is limited to the smallest component. We take the limit as $n\rightarrow \infty$ as the value is inverse proportional to $n$
\[
    \lim_{n\rightarrow\infty}\frac{1}{n}=0
\]
As such
\[
    A=\bigcap_{n=1}^{\infty}A_n=\{0\}
\]
\paragraph{Problem 10}
a)


b)


\paragraph{Problem 11}
As the set is given as an interval it is clear that
\[
    [0,1[\subset\mathbb{R}
\]
As subsets of uncountable sets are uncountable, it becomes clear that the range is uncountable.
\paragraph{Problem 12}
a)

Reading the function it becomes clear that the domain of the function is given by
\[
    \{H,T\}^{3}
\]
While the co-domain is given by
\[
    \mathbb{N}\cup \{0\}
\]
b)

As the function is limited by the amount of heads that can appear in the sequence it is clear that
\[
    \text{Range}(f)=\{0,1,2,3\}
\]
c)

Knowing that $x=2$ tells us that 2 heads are present in the sequence, and as such 1 tails must also be present, as such the possible events are
\[
    \{(H,H,T),(H,T,H),(T,H,H)\}
\]
\paragraph{Problem 13}
a)

We logically assume that the events are disjoint as two teams cant win, as such it is clear that
\[
    0.5+P(b)+0.25=1\Leftrightarrow P(b)=1-0.5-0.25=0.25
\]
b)
As the events are disjoint we determine the probability as
\[
    P(b\cup d)=0.25+0.25=0.5
\]
\paragraph{Problem 14}
a)

By inclusion-exclusion principle we have that
\[
    P(A\cap B)=P(A)+P(B)-P(A\cup B)=0.4+0.7-0.9=0.2
\]
b)

We have that
\[
    P(\overline{A}\cap B)=P(B\setminus A)
\]
Expanding this expression we get that
\[
    P(\overline{A}\cap B)=P(B)-P(A\cap B)=0.7-0.2=0.5
\]
c)

Expanding the expression we again get
\[
    P(A\setminus B)=P(A)-P(A\cap B)=0.4-0.2=0.2
\]
d)

Expanding the expression we get
\[
    P(\overline{A}\setminus B)=P(\overline{A})-P(\overline{A}\cap B)=(1-0.4)-0.5=0.1
\]
e)

By the inclusion-exclusion principle we have
\[
    P(\overline{A}\cup B)=P(\overline{A})+P(B)-P(\overline{A}\cap B)=(1-0.4)+0.7-0.5=0.8
\]
f)
By the distributive law
\[
    A\cap(B\cup\overline{A})=(A\cap B)\cup(A\cap \overline{A})
\]
As
\[
    A\cap\overline{A}=\emptyset
\]
by definition. We get that
\[
    P(A\cap(B\cup\overline{A}))=P((A\cap B)\cup\emptyset)=0.2
\]
\paragraph{Problem 15}
a)

We assume the rolls are independent, as such
\[
    P(X_2=4)=\frac{1}{6}
\]
b)

The sample space is given by
\begin{equation*}
  \begin{gathered}
      S=\{(1,1),(1,2),(1,3),(1,4),(1,5),(1,6),(2,1),(2,2),(2,3),(2,4), \\
      (2,5),(2,6),(3,1),(3,2),(3,3),(3,4),(3,5),(3,6),(4,1),(4,2), \\
      (4,3),(4,4),(4,5),(4,6),(5,1),(5,2),(5,3),(5,4),(5,5), \\
      (5,6),(6,1),(6,2),(6,3),(6,4),(6,5),(6,6)\}$
  \end{gathered}
\end{equation*}
The outcomes that satisfy the event are
\[
    \{X_1+X_2=7\}=\{(1,6),(2,5),(3,4),(4,3),(5,2),(6,1)\}
\]
As such
\[
    P(X_1+X_2=7)=\frac{6}{36}=\frac{1}{6}
\]
c)

As they are independent we determine the probability as
\[
    P(X_1\neq 2\cap X_2\geq 4)=P(X_1\neq 2)P(X_2\geq 4)=\frac{5}{6}\cdot\frac{3}{6}=\frac{15}{36}=\frac{5}{12}
\]
\paragraph{Problem 16}
a)

As the sum of the individual probabilities must equate to 1 we wish to determine a $c$ that satisfies the equation
\[
    \sum_{k=1}^{\infty}P(k)=\sum_{k=1}^{\infty}\frac{c}{3^{k}}=1
\]
We apply the formula for a geometric series
\[
    \sum_{k=0}^{\infty}cr^{k}=\frac{a}{1-r}\text{ for }|r|<1
\]
As this starts from $0$ we subtract $c$ as $P(\{0\}])=\frac{c}{k^{0}}=c$
\begin{align*}
    1&=-c+\sum_{k=0}^{\infty}c\left(\frac{1}{3}\right)^{k} \\
     &=-c+\frac{c}{1-\frac{1}{3}}
\end{align*}
By isolation for $c$ we then get that
\[
    \frac{2}{3}=-\frac{2}{3}c+c\Leftrightarrow \frac{2}{3}=\frac{1}{3}c\Leftrightarrow 2=c
\]
b)

The set is given by the union of the 3, as such
\begin{align*}
    P(\{2,4,6\})&=P(\{2\}\cup \{4\}\cup \{6\}) \\
              &=P(\{2\})+P(\{4\})+P(\{6\}) \\
              &=\frac{2}{9}+\frac{2}{81}+\frac{2}{729} \\
              &=\frac{182}{729}
\end{align*}
c) This must equate to the complement of $P(\{1,2\})$, as such
\begin{align*}
    P(\{3,4,5,\ldots\})&=1-P(\{1,2\}) \\
              &=1-\left(\frac{2}{3}+\frac{2}{9}\right) \\
              &=1-\frac{8}{9} \\
              &=\frac{1}{9}
\end{align*}
\paragraph{Problem 17}
We have that
\begin{align*}
    P(A)&=P(B) \\
    P(C)&=2P(D) \\
    P(A\cup C)&=P(A)+P(C)=0.6
\end{align*}
We rewrite all terms as functions of $P(A)$, these must equate to 1 as a team has to win, making it the sample space
\begin{align*}
    1&=P(A)+P(A)+(0.6-P(A))+\frac{0.6-P(A)}{2} \\
     &=0.5P(A)+0.9
\end{align*}
Isolating for $P(A)$ we get
\[
    P(A)=\frac{1-0.9}{0.5}=0.2
\]
From this we can determine the rest of the probabilities using the requirements set earlier
\begin{align*}
    P(A)&=P(B)=0.2 \\
    P(C)&=0.6-P(A)=0.4 \\
    P(D)&=0.5P(C)=0.2 \\
    \sum P&=0.2+0.2+0.4+0.2=1
\end{align*}
\paragraph{Problem 18}
a)

We insert $t=1$
\[
    P(T\leq 1)=\frac{1}{16}1^{2}=\frac{1}{16}
\]

b) 

This must equate to the complement of it taking 2 hours, as such
\[
    P(2>t)=1-\left(\frac{1}{16}2^{2}\right)=\frac{3}{4}
\]
c)

This equates to the difference between the probability of the job taking more than 1 hour and more than 3 hours
\[
    P(1\leq T\leq 3)=P(T\leq 1)-P(T\leq 3)=\frac{1}{16}3^{2}-\frac{1}{16}1^{2}=\frac{1}{2}
\]
\paragraph{Problem 19}
The problem has the form of the quadratic equation
\[
    ax^{2}+bx+c=0\implies x=\frac{-b\pm\sqrt{b^{2}-4ac}}{2a}
\]
For this to have real solutions it is necessary that
\[
    b^{2}-4ac\geq 0\vee a\neq 0
\]
Translating this to the given equation we wish to determine values $A,B\in[0,1]$ that satisfy
\[
    1^{2}-4AB\geq 0\vee A\neq 0
\]
Seeing this as a function based on the given figure, we write that
\[
    1^{2}-4xy\geq 0
\]
Isolating for $y$ it becomes apparent that
\begin{align*}
    1^{2}-4xy&\geq 0 \\
    1^{2}&\geq 4xy \\
    \frac{1}{4x}&\geq y
\end{align*}
As the maximal $y$-value is 1 we determine the intersection
\[
    1=\frac{1}{4x}\Leftrightarrow x=\frac{1}{4}
\]
Now we can determine the probability as the proportion of the area that fits under this curve in the interval for which it is inside the unit square
\begin{align*}
    P(1-4AB\geq 0)&=\frac{1}{4}+\frac{1}{4}\int_{\frac{1}{4}}^{1}\frac{1}{4x} \\
               &=\frac{1}{4}+\frac{1}{4}\left[\ln{x}\right]_{\frac{1}{4}}^{1} \\
               &=\frac{1}{4}+\frac{1}{4}\left(\ln{1}-\ln{\frac{1}{4}}\right) \\
               &\approx\frac{3}{5}
\end{align*}
\paragraph{Problem 20}
a) 

As every set is a subset of the next, it is clear that $A_n$ will be the largest or equal to the largest set in the sequence, as such
\[
    P\left(\bigcup_{i=1}^{\infty}A_i\right)=\lim_{n\rightarrow \infty}P(A_n)
\]
As all the sets are subsets of $A_n$ and therefore their union will create $A_n$.

b)

Here it is the opposite where $A_n$ will be the smallest or equal to the smallest set in the sequence as such
\[
    P\left(\bigcap_{i=1}^{\infty} A_i\right)=\lim_{n\rightarrow\infty} P(A_n)
\]
As the intersection will always be limited by the smallest set in the sequence.

\paragraph{Problem 21 DO LATER}
\paragraph{Problem 22}
Let $A$ be the event that the customer purchased a cup of coffee and $B$ the event they purchased a piece of cake, then
\begin{align*}
    P(A)&=0.7 \\
    P(B)&=0.4 \\
    P(A\cap B)&=0.2
\end{align*}
By conditional probability we have that
\begin{align*}
    P(A|B)&=\frac{P(A\cap B)}{P(B)} \\
          &=\frac{0.2}{0.4} \\
          &=0.5
\end{align*}
\paragraph{Problem 23}
a)
\begin{align*}
    P(A|B)&=\frac{P(A\cap B)}{P(B)} \\
          &=\frac{0.2}{0.35} \\
          &\approx 0.57
\end{align*}
b)
\begin{align*}
    P(C|B)&=\frac{P(B\cap C)}{P(B)} \\
          &=\frac{0.15}{0.35} \\
          &\approx0.43
\end{align*}
c)
\begin{align*}
    P(B|A\cup C)&=\frac{P(B\cap(A\cup C))}{P(A\cup C)} \\
          &=\frac{0.25}{0.7} \\
          &\approx 0.36
\end{align*}
d)
\begin{align*}
    P(B|A\cap C)&=\frac{P(B\cap(A\cap C))}{P(A\cap C)} \\
          &=\frac{0.1}{0.2} \\
          &=0.5
\end{align*}
\paragraph{Problem 24}
As this is an interval with infinite points, we determine the proportion of the interval that is spanned by the interval were observing, as such
\[
    P(A\leq X\leq B)=\frac{B-A}{10}
\]
a)
\[
    P(2\leq X\leq 5)=\frac{5-2}{10}=0.3
\]
b)
\[
    P(X\leq 2|X\leq 5)=\frac{2}{5}=0.4
\]
c)
\begin{align*}
    P(3\leq X\leq 8|X\geq 4)&=\frac{P(3\leq X\leq 8\cap X\geq 4)}{X\geq 4} \\
                   &=\frac{P(4\leq X\leq 8)}{0.6} \\
                   &=\frac{\frac{8-4}{10}}{0.6} \\
                   &=\frac{2}{3}
\end{align*}
\paragraph{Problem 25}
We define the event $A$ as getting an A in the course and $B$ as living on campus
\begin{align*}
    P(A)&=\frac{120}{600}=0.2 \\
    P(B)&=\frac{200}{600}=\frac{1}{3} \\
    P(A|\overline{B})&=\frac{80}{400}=0.2
    P(A|B)&=\frac{40}{200}=0.2
\end{align*}
For $A$ and $B$ to be independent we would expect
\[
    P(A|B)=P(A)
\]
Which is true, as such, the 2 events are independent.
\paragraph{Problem 26}
a)

Tree is finished on paper.

b) 

We sum the probabilities that result in an error
\begin{align*}
    P(E)&=P(E|G)+P(E\overline{G}) \\
        &=0.08+0.06 \\
        &=0.14
\end{align*}

c)

From the definition of conditional probability we have
\begin{align*}
    P(G|\overline{E})&=\frac{P(G\cap\overline{E})}{P(\overline{E})} \\
                     &=\frac{0.72}{1-0.14} \\
                     &=\frac{0.72}{0.86} \\
                     &\approx 0.84
\end{align*}
\paragraph{Problem 27}
\paragraph{Problem 28}
We define $D$ as the unit being defective, the probability of picking a defective unit on the first draw is
\[
    P(D_1)=\frac{5}{100}
\]
Assuming that one wasnt defective the next draw has
\[
    P(D_2|D_1)=\frac{5}{99}
\]
And the 3rd
\[
    P(D_3|D_2,D_1)=\frac{5}{98}
\]
We wish to determine the probaility of the event
\[
    P(D=1)=P(\{(D,\overline{D},\overline{D}),(\overline{D},D,\overline{D}),(\overline{D},\overline{D},D)\})
\]
As these are disjoint we determine them as the sum of their individual probailities
\begin{align*}
P(D=1)&=P(\{(D,\overline{D},\overline{D})\})+P(\{\overline{D},D,\overline{D}\})+P(\{\overline{D},\overline{D},D)\}) \\
      &=\frac{5}{100}\cdot\frac{95}{99}\cdot\frac{94}{98}+\frac{95}{100}\cdot\frac{5}{99}\cdot\frac{94}{98}+\frac{95}{100}\cdot\frac{94}{99}\cdot\frac{5}{98} \\
      &\approx 0.14
\end{align*}
\paragraph{Problem 29}
\paragraph{Problem 30}
\paragraph{Problem 31}
\paragraph{Problem 32}
\paragraph{Problem 33}
\paragraph{Problem 34}
\paragraph{Problem 35}
\paragraph{Problem 36}
\paragraph{Problem 37}
\paragraph{Problem 38}
\paragraph{Problem 39}
