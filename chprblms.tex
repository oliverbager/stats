\subsection{Chapter problems}
\subsubsection{Chapter 1}
\paragraph{Problem 1}
Suppose the universal set is defined as $S=\{x\in \mathbb{N} | 1\geq x \geq 10\}$, $A=\{1,2,3\}$, $B=\{x\in S|2\geq x\geq 7\}$ and $C=\{7,8,9,10\}$.

a) Find $A\cup B$
\[
    A\cup B=\{1,2,3\}\cup\{2,3,4,5,6,7\}=\{1,2,3,4,5,6,7\}
\]
b) Find $(A\cup C)\setminus B$
\[
    (A\cup C)\setminus B=\{1,2,3,7,8,9,10\}\setminus \{2,3,4,5,6,7\}=\{1,8,9,10\}
\]
c) Find $\overline{A}\cup(B\setminus C)$
\[
    \overline{A}\cup(B\setminus C)=\{4,5,6,7,8,9,10\}\cup\{2,3,4,5,6\}=\{2,3,4,5,6,7,8,9,10\}
\]
d) Do $A,B$ and $C$ form a partition of $S$?

No as $A\cap B\neq\emptyset$ and $B\cap C\neq\emptyset$.
\paragraph{Problem 2}
When working with real numbers, our universal set is $\mathbb{R}$. Find each of the following sets
a) $[6,8]\cup[2,7[$
\[
    [2,8]
\]
b) $[6,8]\cap[2,7[$
\[
    [6,7[
\]
c) $\overline{[0,1]}$
\[
    ]-\infty,0[\cup]1,\infty[
\]
d) $[6,8]\setminus ]2,7[$
\[
    [7,8]
\]
\paragraph{Problem 3}
For each of the follopwing Venn diagrams, write the sets denoted by the shaded area

a)
\[
    A\cup B\setminus A\cap B
\]
b)
\[
    B\setminus C
\]
c)
\[
    A\cap B+A\cap C
\]
d)
\[
    C\cup(A\cap B)\setminus ((C\cap A)\cup (C\cap B))
\]
\paragraph{Problem 4}
A coin is tossed twice. Let $S=\{H,T\}\times\{H,T\}$. Write the following sets by listing their elements:

a) First toss is heads
\[
    A=\{(H,T),(H,H)\}
\]
b) At least one tails
\[
    B=\{(T,T),(T,H),(H,T)\}
\]
c) Two tosses are different
\[
    C=\{(H,T),(T,H)\}
\]
\paragraph{Problem 5}
Let $A=\{1,2,\ldots,100\}$. For any $i\in \mathbb{N}$, define $A_{i}$ as the set of numbers in $A$ that are divisible by $i$.

a)
\begin{align*}
    |A_{1}|&=\left\lfloor{\frac{|S|}{1}}\right\rfloor=\left\lfloor{\frac{100}{1}}\right\rfloor=100 \\
    |A_{2}|&=\left\lfloor{\frac{|S|}{2}}\right\rfloor=\left\lfloor{\frac{100}{2}}\right\rfloor=50 \\
    |A_{3}|&=\left\lfloor{\frac{|S|}{3}}\right\rfloor=\left\lfloor{\frac{100}{3}}\right\rfloor=33 \\
    |A_{4}|&=\left\lfloor{\frac{|S|}{4}}\right\rfloor=\left\lfloor{\frac{100}{4}}\right\rfloor=25 \\
    |A_{5}|&=\left\lfloor{\frac{|S|}{5}}\right\rfloor=\left\lfloor{\frac{100}{5}}\right\rfloor=20
\end{align*}
b)

\paragraph{Problem 6}
\paragraph{Problem 7}
\paragraph{Problem 8}
\paragraph{Problem 9}
\paragraph{Problem 10}
\paragraph{Problem 11}
\paragraph{Problem 12}
\paragraph{Problem 13}
\paragraph{Problem 14}
\paragraph{Problem 15}
\paragraph{Problem 16}
\paragraph{Problem 17}
\paragraph{Problem 18}
\paragraph{Problem 19}
\paragraph{Problem 20}
\paragraph{Problem 21}
\paragraph{Problem 22}
\paragraph{Problem 23}
\paragraph{Problem 24}
\paragraph{Problem 25}
\paragraph{Problem 26}
\paragraph{Problem 27}
\paragraph{Problem 28}
\paragraph{Problem 29}
\paragraph{Problem 30}
\paragraph{Problem 31}
\paragraph{Problem 32}
\paragraph{Problem 33}
\paragraph{Problem 34}
\paragraph{Problem 35}
\paragraph{Problem 36}
\paragraph{Problem 37}
\paragraph{Problem 38}
\paragraph{Problem 39}
