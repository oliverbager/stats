\section{Chapter 3}
\subsection{Random variables}
\begin{definition}[Random variable]
    A random variable is a real-valued value determined by an underlying random experiment. Mathematically this can be expressed as a function that assigns a value to a possible outcome, $S\xrightarrow{X} \mathbb{R}$.
\end{definition}
\subsubsection{Discrete random variables}
What defines whether at random variable is discrete, continuous or mixed is it's range. If the range of the random variable is strictly countable, the random variable is said to be discrete, this for example occurs if we wish to determine the amount of heads in a coin throw.

\subsubsection{Probability mass function}
Let $X$ be a discrete random variable, as such
\[
    R_X=\{x_{1},x_{2},x_{3},\ldots\}
\]
As the possible outcomes are discrete, they must each have their own probability of occuring, we define the event
\[
    A=\{s\in S|X(s)=x_{i}\}
\]
Where the probabilities of each individual event $\{X=x_{i}\}$ is given by the probability mass function of $X$
\[
    P_{X}(x_{i})=P(X=x_{i})\text{, for} i=1,2,3,\ldots
\]
To ensure the function is well-defined, the definition of the mass probability function is frequently extended to all real numbers by a piecewise function
\[
    P_{X}(x)=\begin{cases}P(X=x) & x\in R_X \\ 0 & x\notin R_X\end{cases}
\]
As this is a plottable discrete function, this is also called the probability mass function of the random variable $X$.

\subsubsection{Independent random variables}
Random variables can, like probabilities and conditional probabilities, be independent. Independence is defined in a similar way to in those examples, 2 random variables, $X$ and $Y$, are independent if
\[
    P(X=x\cap Y=y)=P(X=x)P(Y=y) \text{, for all } x,y
\]
This means that if they are independent we can write
\[
    P(X\in A\cap Y\in B)=P(X\in A)P(Y\in B)\text{, for all sets } A \text{ and } B
\]
We can also condition random variables on eachother, for two independent random variables it then follows that
\[
    P(X=x|Y=y)=P(X=x)\text{, for all } x,y
\]
As knowing the outcome of $Y$ does not impact the probability of $X$.

\subsection{Special distributions}
\subsubsection{Bernoulli distribution}
We recall that a Bernoulli experiment is an experiment with only two possible outcomes, success or failure, defining $x=1$ as success and $x=0$ as failure, we get the Bernoulli PMF
\[
    X\sim Bernoulli(p)=P_{X}(k)=\begin{cases}p & x=1 \\ 1-p & x=0 \\ 0 & \text{otherwise}\end{cases}
\]
\subsubsection{Geometric distribution}
For a geometric experiment, that being a Bernoulli experiment where we repeat the experiment until one of the two outcomes is observed has the probability $P(A)=p(1-p)^{k-1}$, where $p$ is the probability of observing what we want, as such the PMF is given by
\[
    X\sim Geometric(p)=P_{X}(k)=\begin{cases}p(1-p)^{k-1} & k\in \mathbb{N} \\ 0 & \text{otherwise}\end{cases}
\]
\subsubsection{Binomial distribution}
We recall that the binomial theorem states that
\[
    P(k)=\begin{pmatrix}n\\k\end{pmatrix}p^{k}(1-p)^{n-k}
\]
As such the PMF is given by
\[
    X\sim Binomial(n,p)=P_{X}(k)=\begin{cases}\begin{pmatrix}n\\k\end{pmatrix}p^{k}(1-p)^{n-k} & k\in\{0,1,2,\ldots,n\} \\ 0 & \text{otherwise}\end{cases}
\]
Simultaneously we can observe each cointoss that makes up the sequence as individual Bernoulli distributions, as such we can express a binomial as the sum of a sequence of Bernoulli experiments, i.e. we have the PMF $X$ defined as the binomial giving the number of heads, which can be expressed as
\[
    X=\sum_{i=1}^{n}X_{i}
\]
Where $i$ is the $i$th cointoss in the sequence as given by the Bernoulli PMF.
\begin{theorem}
    Given two independent random variables described by binomial distributions, $X,Y$, with $k_{1}=n$ and $k_{2}=m$ the random variable defined as $Z=X+Y$ is given by
    \[
        P_{Z}(k)=\begin{cases}\begin{pmatrix}n+m\\k\end{pmatrix}p^{k}(1-p)^{n+m-k} & k=\{0,1,2,\ldots,m+n\} \\ 0 & \text{otherwise}\end{cases}
    \]
\end{theorem}
\begin{proof}
  As the random variable can be described as the sum of the individual Bernoulli random variables we rewrite that
  \begin{align*}
      Z&=X+Y \\
       &=X_{1}+X_{2}+\ldots+X_{n}+Y_{1}+Y_{2}+\ldots+Y_{n}
  \end{align*}
  A way to express this as probabilities is
  \begin{align*}
      P_{Z}(k)&=P(Z=k) \\
              &=P(X+Y=k) \\
              &=\sum_{i=0}^{n}P(X+Y=k|X=i)P(X=i) \\
              &=\sum_{i=0}^{n}P(X=k-i)P(X=i) \\
              &=\sum_{i=0}^{n}\begin{pmatrix}m\\k-i\end{pmatrix}p^{k-i}(1-p)^{m-k+i}\begin{pmatrix}n\\i\end{pmatrix}p^{i}(1-p)^{n-i} \\
              &=\sum_{i=0}^{n}\begin{pmatrix}m\\k-i\end{pmatrix}\begin{pmatrix}n\\i\end{pmatrix}p^{k}(1-p)^{m-k+n} \\
              &=p^{k}(1-p)^{m-k+n}\sum_{i=0}^{n}\begin{pmatrix}m\\k-i\end{pmatrix}\begin{pmatrix}n\\i\end{pmatrix} \\
              &=\begin{pmatrix}m+n\\k\end{pmatrix}p^{k}(1-p)^{m-k+n} \\
              &=\begin{cases}\begin{pmatrix}m+n\\k\end{pmatrix}p^{k}(1-p)^{m-k+n} & k\in\{0,1,2,\ldots,m+n\} \\ 0 & \text{otherwise}\end{cases}
  \end{align*}
\end{proof}
\subsubsection{Negative binomial (Pascal) distribution}
The Pascal distribution is a generalization of the geometric distribution where we wish to observe $m$ successes instead of just $1$, by this it follows that a Pascal distribution with $m=1$ is simply a geometric distribution.

Suppose we toss a coin until $m$ heads are observed, where $X$ is the total number of tosses, in this context we define the event $A=\{X=k\}$ as the event that $m$ heads are observed in $k$ tosses. We rewrite this event as the intersection $A=B\cap C$ where
\begin{itemize}
    \item[-] $B$ is the event that $m-1$ heads are observed in the first $k-1$ tosses.
    \item[-] $C$ is the event that heads is observed in the $k$th toss.
\end{itemize}
As these two events are independent (seperate coin tosses), we have that
\[
    P(A)=P(B\cap C)=P(B)P(C)
\]
We can determine these probabilities. $B$ is given by a binomial distribution whilst $C$ is a constant value, as such we write that
\begin{align*}
    P(B)&=\begin{pmatrix}k-1\\m-1\end{pmatrix}p^{m-1}(1-p)^{(k-1)-(m-1)} \\
        &=\begin{pmatrix}k-1\\m-1\end{pmatrix}p^{m-1}(1-p)^{k-m} \\
    P(C)&=p
\end{align*}
Since $A$ is given by the product of these, we multiply them together whereby we get
\begin{align*}
    P(A)&=P(B)P(C) \\
        &=\begin{pmatrix}k-1\\m-1\end{pmatrix}p^{m-1}(1-p)^{k-m}p \\
        &=\begin{pmatrix}k-1\\m-1\end{pmatrix}p^{m}(1-p)^{k-m}
\end{align*}
Whereby we can write the PMF for the Pascal distribution as
\[
    X\sim Pascal(m,p)=P_{X}(k)=\begin{cases}\begin{pmatrix}k-1\\m-1\end{pmatrix}p^{m}(1-p)^{k-m} & k\in\{m,m+1,m+2,\ldots\} \\ 0 & \text{otherwise}\end{cases}
\]
\subsubsection{Hypergeometric distribution}
Imagine a bag containing $b$ blue marbles and $r$ red ones. Choosing $k\leq b+r$ marbles at random without replacements will at best give us $X\leq \text{min}(k,b)$ blue marbles, whilst the number of red marbles in our sample must be $X\geq \text{max}(0,k-r)$. As such we can define the range of $X$ as
\[
    R_X=\{\text{max}(0,k-r),\text{max}(0,k-r)+1,\ldots,\text{min}(k,b)\}
\]
Simultaneously we can create an expression for determining the probability of getting $x$ blue marbles (and $k-x$ red ones) by
\[
    \frac{\begin{pmatrix}b\\x\end{pmatrix}\begin{pmatrix}r\\k-x\end{pmatrix}}{\begin{pmatrix}b+r\\k\end{pmatrix}} \text{, for } x\in R_X
\]
Using this expression we create the PMF as
\[
    X\sim Hypergeometric(b,r,k)=P_{X}(x)=\begin{cases}\frac{\begin{pmatrix}b\\x\end{pmatrix}\begin{pmatrix}r\\k-x\end{pmatrix}}{\begin{pmatrix}b+r\\k\end{pmatrix}} & x\in R_X \\ 0 & \text{otherwise}\end{cases}
\]
\subsubsection{Poisson distribution}
The Poisson distribution is tupically used to model occurences in an interval using a variable, $\lambda$, as the known average.
\begin{theorem}
    Let $X$ be a binomial PMF with $p=\frac{\lambda}{n},\lambda>0$. Then for any $k\in\{0,1,2,\ldots\}$ we have that $\lim_{n\rightarrow\infty}P_{X}(k)=\frac{e^{-\lambda}\lambda^{k}}{k!}$.
\end{theorem}
\begin{proof}
  The binomial will be given by
  \[
    P_{X}(k)=\begin{pmatrix}n\\k\end{pmatrix}\left(\frac{\lambda}{n}\right)^{k}\left(1-\frac{\lambda}{n}\right)^{n-k}
  \]
  Taking the limit of this binomial we get
  \begin{align*}
      \lim_{n\rightarrow\infty}P_{X}(k)&=\lim_{n\rightarrow\infty}\begin{pmatrix}n\\k\end{pmatrix}\left(\frac{\lambda}{n}\right)^{k}\left(1-\frac{\lambda}{n}\right)^{n-k} \\
                        &=\lambda^{k}\lim_{n\rightarrow\infty}\frac{n!}{k!(n-k)!}\left(\frac{1}{n^{k}}\right)\left(1-\frac{\lambda}{n}\right)^{n-k} \\
                        &=\frac{\lambda^{k}}{k!}\lim_{n\rightarrow\infty}\frac{n!}{(n-k)!}\left(\frac{1}{n^{k}}\right)\left(1-\frac{\lambda}{n}\right)^{n}\left(1-\frac{\lambda}{n}\right)^{-k} \\
                        &=\frac{\lambda^{k}}{k!}\lim_{n\rightarrow\infty}\left(\frac{n(n-1)\ldots(n-k+1)}{n^{k}}\right)\left(1-\frac{\lambda}{n}\right)^{n}\left(1-\frac{\lambda}{n}\right)^{-k} \\
  \end{align*}
  We now take the limit of each of these expressions individually
  \begin{align*}
      \lim_{n\rightarrow\infty}\frac{n(n-1)\ldots(n-k+1)}{n^{k}}&=1 \\
      \lim_{n\rightarrow\infty}\left(1-\frac{\lambda}{n}\right)^{n}&=e^{-\lambda} \\
      \lim_{n\rightarrow\infty}\left(1-\frac{\lambda}{n}\right)^{-k}&=1 
  \end{align*}
  Which gives us
  \begin{align*}
      P_{X}(k)&=\frac{\lambda^{k}}{k!}\cdot 1\cdot 1\cdot e^{-\lambda} \\
              &=\frac{e^{-\lambda}\lambda^{k}}{k!}
  \end{align*}
\end{proof}
As such we can define the Poisson distributions PMF as
\[
    X\sim Poisson(\lambda)=P_{X}(k)=\begin{cases}\frac{e^{-\lambda}\lambda^{k}}{k!} & k\in R_X \\ 0 & \text{otherwise}\end{cases}
\]
\subsection{Cumulative distribution function}
As opposed to the discrete distribution function the cumulative distribution function describes the probability distribution on a continuous line.
\begin{definition}
  The cumumlative distribution functions of a random variable $X$ is defined as $F_{X}(x)=P(X\leq x)\text{, for all } x\in \mathbb{R}$.
\end{definition}
A CDF can also be made for discrete random variables, this is done by extrapolating the range for which a probability is true to the real line, see for example a coin toss given by a binomial PMF with $p=\frac{1}{2},n=2$
\begin{align*}
    P_{X}(k)&=\begin{pmatrix}n\\k\end{pmatrix}\left(\frac{1}{2}\right)^{k}\left(1-\frac{1}{2}\right)^{n-k} \\
    P_{X}(0)&=\frac{2!}{0!(2-0)!}\cdot 1\cdot\frac{1}{4}=\frac{1}{4} \\
    P_{X}(1)&=\frac{2!}{1!(2-1)!}\cdot\frac{1}{2}\cdot\frac{1}{2}=\frac{1}{2} \\
    P_{X}(2)&=\frac{2!}{2!(2-2)!}\cdot\frac{1}{4}\cdot 1=\frac{1}{4}
\end{align*}
To convert this to a CDF we extrapolate the range for which the probabilities are valid
\[
    F_{X}(x)=\begin{cases}0 & x<0 \\ \frac{1}{4} & x\leq x<1 \\ \frac{3}{4} & 1\leq x<2 \\ 1 & x\geq 2 \end{cases}
\]
Here we see that the CDF is equal to summing up the probabilities gradually
\[
    F_{X}(x)=\sum_{x_{k}\leq x}P_{X}(x_{k})
\]
\begin{theorem}
  For all $a\leq b$ we have that $P(a<X\leq b)=F_{X}(b)-F_{X}(a)$.
\end{theorem}
\begin{proof}
    For $a\leq b$ we have that
    \[
        P(X\leq b)=P(x\leq a)+P(a<X\leq b)
    \]
    Rewriting this as CDFs we have that
    \[
        F_{X}(b)=F_{X}(a)+P(a<X\leq b)
    \]
    Using simple algebraic manipulation we then get
    \[
        P(a<X\leq b)=F_{X}(b)-F_{X}(a)
    \]
\end{proof}
\subsubsection{Expectation}
A random variable has an expected value, also known as othe mean, which is the weighted average of all values in the range.
\begin{definition}
    Let $X$ be a discrete random variable with $R_X=\{x_{1},x_{2},\ldots\}$ (finite or countably infinite), the expected value of $X$ denoted $EX$ is then defined as
    \[
        EX=\sum_{x_{k}\in R_X}x_{k}P(X=X_{k})=\sum_{x_{k}\in R_X}x_{k}P_{X}(x_{k})
    \]
\end{definition}
The expected value of a random variable is the value which we expect the mean to take on as the number of trials approaches infinity, other ways to denote the expected value is
\[
    EX=E[X]=E(X)=\mu_{X}
\]
This lets us establish models for the expected value of different known distribution functions

\paragraph{Bernoulli expected value}
The expected value of the Bernoulli is given by
\begin{align*}
    EX&=0P_{X}(0)+1P_{X}(1) \\
      &=0\cdot(1-p)+1\cdot(p) \\
      &=p
\end{align*}
\paragraph{Geometric expected value}
The expected value of the geometric is given by
\begin{align*}
    EX&=\sum_{x_{k}\in R_{X}}x_{k}P_{X}(x_{k}) \\
      &=\sum_{k=1}^{\infty}kq^{k-1}p \\ 
      &=p\sum_{k=1}^{\infty}kq^{k-1}
\end{align*}
We make use of the geometric sum formula to determine the value of this
\[
    \sum_{k=0}^{\infty}x^{k}=\frac{1}{1-x}\text{ for }|x|<1
\]
Our sum is however of a different form, however we realise we can take the derivative with respect to $x$ on both sides and get
\begin{align*}
    \frac{d}{dx}\sum_{k=0}^{\infty}x^{k}&=\frac{d}{dx}\frac{1}{1-x} \\
    \sum_{k=0}^{\infty}kx^{k-1}&=\frac{1}{(1-x)^{2}}
\end{align*}
As such we can write that
\begin{align*}
    EX&=p\sum_{k=1}^{\infty}kq^{k-1} \\
      &=p\frac{1}{(1-q)^{2}} \\
      &=p\frac{1}{p^{2}} \\
      &=\frac{1}{p}
\end{align*}
\paragraph{Poisson expected value}
The expected value of the Poisson is given by
\begin{align*}
    EX&=\sum_{x_{k}\in R_{X}}x_{k}P_{X}(x_{k}) \\
      &=\sum_{k=0}^{\infty}k\frac{e^{-\lambda}\lambda^{k}}{k!} \\
      &=\sum_{k=0}^{\infty}\frac{e^{-\lambda}\lambda^{k}}{(k-1)!} \\
\end{align*}
To allow a favourable rewrite we now define $j=k-1$, as such
\begin{align*}
    EX&=\sum_{k=0}^{\infty}\frac{e^{-\lambda}\lambda^{j+1}}{j!} \\
      &=\lambda e^{-\lambda}\sum_{j=0}^{\infty}\frac{\lambda^{j}}{j!}
\end{align*}
As this is the Taylor series for $e^{\lambda}$ we get
\begin{align*}
    EX&=\lambda e^{-\lambda}\sum_{j=0}^{\infty}\frac{\lambda^{j}}{j!} \\
      &=\lambda e^{-\lambda}e^{\lambda} \\
      &=\lambda
\end{align*}
An important concept in expectation is linearity, which is useful for calculating the expected value of linear functions of random variables. The theorem states that
\begin{itemize}
    \item[-] $E[aX+b]=aEX+b$, for all $a,b\in\mathbb{R}$
    \item[-] $E[X_{1}+X_{2}+\ldots+X_{n}]=EX_{2}+EX_{1}+\ldots+EX_{n}$, for any set of random variables.
\end{itemize}
\paragraph{Binomial expected value}
As demonstrated earlier, the binomial can be seen as a sum of Bernoulli random variables, as we know the expected value of the Bernoulli, we make use of linearity
\begin{align*}
    EX&=\begin{pmatrix}n\\k\end{pmatrix}p^{k}(1-p)^{n-k} \\
      &=EX_{1}+EX_{2}+\ldots+EX_{n} \\
      &=p+p+\ldots+p \\
      &=np
\end{align*}
\paragraph{Pascal expected value}
As in the binomial we can see the pascal as the sum of a sequence of geometric random variables, as such we again have by linearity that
\begin{align*}
    EX&=\begin{pmatrix}n-1\\k-1\end{pmatrix}p^{k-1}(1-p)^{n-k} \\
      &=EX_{1}+EX_{2}+\ldots+EX_{n} \\
      &=\frac{1}{p}+\frac{1}{p}+\ldots+\frac{1}{p} \\
      &=\frac{n}{p}
\end{align*}
\subsection{Functions of random variables}
If $X$ is a random variable and $Y=g(X)$ then $Y$ is a random variable. As such we can talk about its PMF, CDF and $EX$. The range of $Y$ can be determined as
\[
    R_{Y}=\{g(x)|x\in R_{X}\}
\]
This also means that we can determine the expected value of a function of a random variable. We let $X$ with PMF $P_{X}(x)$ be a discrete random variable and $Y=g(X)$. One way to determine $EY$ is to determine the PMF of $Y$ and then applying the expectation formula, however the law of the unconscious statisticatian which states that
\[
    E[g(X)]=\sum_{x_{k}\in R_{X}}g(x_{k})P_{X}(x_{k})
\]
\begin{theorem}
    The expected value of a function of a discrete random variable is given by the law of the unconscious statistician, $E[g(X)]=\sum_{k_{k}\in R_{X}}g(x_{k})P_{X}(x_{k})$.
\end{theorem}
\begin{proof}
    By definition we have that
    \[
        EY=\sum_{y_{k}\in R_{Y}}y_{k}P_{Y}(y_{k})
    \]
    As $Y=g(X)$ we can rewrite this expression as
    \[
        E[g(X)]=\sum_{x_{k}\in R_{X}}x_{k}P_{X}(x_{k})
    \]
\end{proof}
\subsubsection{Variance}
The expected value however has a problem, as it ignores the form of the distribution, two PMFs with the same expected value may have very different distributions. This is represented by the variance of the PMF, which denotes how spread out the distribution is.
\begin{definition}
  The variance of a random variable $X$ with $EX=\mu_{X}$ is defined as
  \[
      \text{Var}(X)=E[(X-\mu_{X})^{2}]
  \]
\end{definition}
This works, as the variability will be high if $X$ is often far away from the mean, $\mu_{X}$, whilst it'll be low if it is close to the mean, we can therefore use the variance as a measure of how spread the distribution is.

To compute the variance we simply use LOTUS with $g(X)=(X-\mu_{X})^{2}$
\[
    \text{Var}(X)=E[(X-\mu_{X})^{2}]=\sum_{x_{k}\in R_{X}}(x_{k}-\mu_{X})^{2}P_{X}(x_{k})
\]
\begin{theorem}
    The variance of a random variable $X$ can be computed as $\text{Var}(X)=E[X² ]-[EX]^{2}$.
\end{theorem}
\begin{proof}
    By the definition of variance we have that
    \begin{align*}
        \text{Var}(X)&=E[(X-\mu_{X})^{2}] \\
                 &=E[X^{2}+\mu_{X}^{2}-2X\mu_{X}]
    \end{align*}
    By linearity we then have that
    \begin{align*}
        \text{Var}(X)&=E[X^{2}]+E[\mu_{X}]-E[2X\mu_{X}] \\
                 &=E[X^{2}]+\mu_{X}^{2}-2\mu_{X}E[X] \\
                 &=E[X^{2}]+\mu_{X}^{2}-2\mu_{X}^{2} \\
                 &=E[X^{2}]-\mu_{X}^{2} \\
                 &=\left[\sum_{x_{k}\in R_{x}}x_{k}^{2}P_{X}(x_{k})\right]-\mu_{X}^{2}
    \end{align*}
\end{proof}
One problem with the variance is that it ends up being the wrong unit as the value is squared, as such we define the standard deviation as the square root of the variance
\[
    \text{SD}(X)=\sigma_{X}=\sqrt{\text{Var}(X)}
\]
\begin{theorem}
  For a random variable $X$ and real numbers $a,b$ we have that $\text{Var}(aX+b)=a^{2}\text{Var}(X)$.
\end{theorem}
\begin{proof}
  If $Y=aX+b$ then
  \[
      \text{Var}(Y)&=E[(Y-EY)^{2}] \\
  \]
  By linearity we have that
  \[
      \text{Var}(Y)=E[(aX+b-aEX-b)^{2}] \\
  \]
  Reducing and factoring $a$
  \begin{align*}
      \text{Var}(Y)&=E[(aX-aEX)^{2}] \\
               &=E[(a^{2}(X-EX)^{2})] \\
  \end{align*}
  As $a$ is a constant we have that
  \begin{align*}
      \text{Var}(Y)&=a^{2}E[(X-\mu_{X})^{2}] \\
               &=a^{2}\text{Var}(X)
  \end{align*}
\end{proof}
\begin{theorem}
  For a random variable $X$ and real numbers $a,b$ we have that $\text{SD}(aX+b)=|a|\text{SD}(X)$.
\end{theorem}
\begin{proof}
  As
  \[
      \text{SD}(X)=\sqrt{\text{Var(X)}}
  \]
  We square both sides of the previous theorem
  \begin{align*}
      \sqrt{\text{Var(Y)}}&=\sqrt{a^{2}\text{Var}(X)} \\
                  &=\sqrt{a^{2}}\sqrt{\text{Var}(X)}
  \end{align*}
  This is equivalent to
  \[
      \text{SD}(Y)=|a|\text{SD}(X)
  \]
\end{proof}
