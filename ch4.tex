\section{Chapter 4}
\subsection{Continuous random variables and their distributions}
Suppose we have an interval on the real line, $[a,b]$ with a uniformly distributed probability of a given value being picked, we know that the probability of each point $P(X=x)=0$ as there is an infinite amount of points, as such, it only makes sense to look at subintervals of the interval when it comes to probability. 

Simply from the definition of a CDF we have that $F_{X}(x<a)=0$, whilst $F_{X}(x\geq b)=1$, simultaneously we can establish a general equation for the probability of an interval as the proportion it constitutes of the total interval
\begin{align*}
    F_{X}(a\leq x_{1}\leq x_{2}\leq b)&=P(X\in[x_{1},x_{2}]) \\
                       &=\frac{x_{2}-x_{1}}{b-a}
\end{align*}
Using this definition we can create a CDF as
\[
    F_{X}(x)\begin{cases}0 & x<a \\ \frac{x-a}{b-a} & a\leq x\leq b \\ 1 & x\geq b\end{cases}
    
\]
Whether we use $<$ or $\leq$ (or the reverse), doesn't matter as the probability of each individual point is equal to 0, as such $P(X<2)=P(X\leq 2)$.
\begin{definition}
  A random variable $X$ with CDF $F_{X}(x)$ is said to be continuous if $F_{X}(x)$ is a continuous function for all $x\in \mathbb{R}$.
\end{definition}
\subsection{Probability density function}
As its impossible to define a PMF for a continuous function (as $P(X=x)=0$ for all $x\in \mathbb{R}$), as instead define the probability density function as 
\[
    f_{X}(x)=\lim_{\Delta\rightarrow0^{+}}\frac{P(x<X\leq x+\Delta)}{\Delta}
\]
We recall that $P(a<X\leq b)=F_{X}(b)-F_{X}(a)$, as such we rewrite the limit as
\[
    f_{X}(x)=\lim_{\Delta\rightarrow0^{+}}\frac{F_{X}(x+\Delta)-F_{X}(x)}{\Delta}
\]
We then recognize this as the definition of the derivate, as such we can write that
\[
    f_{X}(x)&=\lim_{\Delta\rightarrow0^{+}}\frac{F_{X}(x+\Delta)-F_{X}(x)}{\Delta}=F'_{X}(x)
\]
\begin{definition}
  Consider a continuous random variable $X$ with an absolutely continuous CDF $F_{X}(x)$. The function $f_{X}(x)$ defined by
  \[
  f_{X}(x)=\frac{d}{dx}F_{X}(x)=F'_{X}(x)$ if $F_{X}(x) 
  \]
  is differentiable at $x$ is called the probability density function of $X$.
\end{definition}
Using the general example from the previous section we determine the PDF as
\begin{align*}
    F'_{X}(x)&=\frac{d}{dx}\frac{x-a}{b-a} \\
             &=\frac{1}{b-a}\left(\frac{d}{dx}x+\frac{d}{dx}-a\right) \\
             &=\frac{1}{b-a}
\end{align*}
As such we can define the probability density function as
\[
    f_{X}(X)=\begin{cases}\frac{1}{b-a} & a<x<b \\ 0 & x\notin [a,b]\end{cases}
\]
As the PDF is the derivative of the CDF, we can integrate a segment of the interval and get the probability thereof assuming it is absolutely continuous in that interval
\[
    F_{X}(x)=\int_{-\infty}^{x}f_{X}(x)dx
\]
At the same time the integral over the entire real line must be equal to $1$ in accordance with the axioms of probability
\[
    \int_{-\infty}^{\infty}f_{X}(x)dx=1
\]
Whilst we can determine the probability of an interval as
\[
    P(a<X\leq b)=F_{X}(b)-F_{X}(a)=\int_{a}^{b}f_{X}(x)dx
\]
Similarly we can define the range of a random variable $X$ as the possible values of the random varibles where the PDF is larger than $0$, as such
\[
    R_{X}=\{x|f_{X}(x)>0\}
\]
\subsection{Expected value and variance}
In discrete random variables we frequently used sums to determine different values, in the case of continuous random variables we replace the sum with an integral sign and the PMF with the PDF, using LOTUS as an example we for example get that
\begin{align*}
    EX_{discrete}&=\sum_{x_{k}\in R_{X}}x_{k}P_{X}(x_{k}) \\
    EX_{continuous}&=\int_{-\infty}^{\infty}xf_{X}(x)dx
\end{align*}
For variance we can do the same as
\begin{align*}
    \text{Var}(X)&=E[(X-\mu_{X})^{2}]\\
             &=EX^{2}-(EX)^{2} \\
             &=\int_{-\infty}^{\infty}x^{2}f_{X}(x)dx-\mu_{X}^{2}
\end{align*}
\subsection{Functions of continuous random variables}
If $X$ is a continuous random variable and $Y=g(X)$ is a function of $X$, then $Y$ is a random variable, as such finding the CDF and PDF of $Y$ should be possible, either directly or by determining CDF and taking the derivative, to do so we however have to make sure the function is continuous over the real line.

\subsubsection{Uniform distribution}
A uniform distribution is a distribution with a PDF given by
\[
    f_{X}(x)=\begin{cases}\frac{1}{b-a} & a<x<b \\ 0 & x\notin[a,b]\end{cases}
\]
The expected value of a uniform distribution is given by
\begin{align*}
    \text{Var}(x)&=\int_{-\infty}^{\infty}xf_{X}(x)dx \\
             &=\int_{-\infty}^{\infty}x\times\left(\frac{1}{b-a}\right)dx \\
             &=\frac{1}{b-a}\times\left[\frac{x^{2}}{2}\right]_{a}^{b} \\
             &=\frac{1}{b-a}\times\left(\frac{b^{2}}{2}-\frac{a^{2}}{2}\right) \\
             &=\frac{b^{2}-a^{2}}{2(b-a)} \\
             &=\frac{b-a}{2}
\end{align*}
To determine the variance we need $EX^{2}$
\begin{align*}
    EX^{2}&=\int_{-\infty}^{\infty}x^{2}f_{X}(x)dx \\
       &=\int_{-\infty}^{\infty}x^{2}\times\left(\frac{1}{b-a}\right)dx \\
       &=\frac{1}{b-a}\times\left[\frac{x^{3}}{3}\right]_{a}^{b} \\
       &=\frac{1}{b-a}\times\left(\frac{b^{3}}{3}-\frac{a^{3}}{3}\right) \\
       &=\frac{b^{3}-a^{3}}{3(b-a)} \\
       &=\frac{(b-a)(b^{2}+a^{2}+ab)}{3(b-a)} \\
       &=\frac{b^{2}+a^{2}+ab}{3}
\end{align*}
As such we can determine the variance as
\begin{align*}
    Var(X)&=EX^{2}-(EX)^{2} \\
          &=\frac{b^{2}+a^{2}+ab}{3}-\left(\frac{b-a}{2}\right)^{2} \\
          &=\frac{b^{2}+a^{2}+ab}{3}-\frac{(b-a)^{2}}{4} \\
          &=\frac{4(b^{2}+a^{2}+ab)}{12}-\frac{3(b^{2}+a^{2}-2ab)}{12} \\
          &=\frac{4b^{2}+4a^{2}+4ab-3b^{2}-3a^{2}+6ab}{12} \\
          &=\frac{b^{2}+a^{2}-2ab}{12}
\end{align*}
\subsubsection{Exponential distribution}
An exponential distribution is a distribution with a PDF given by
\[
    f_{X}(x)=\begin{cases}\lambda e^{-\lambda x} & x>0 \\ 0 & \text{otherwise}\end{cases}
\]
The expected value of a uniform distribution is given by
\begin{align*}
    EX&=\int_{-\infty}^{\infty}xf_{X}(x)dx \\
      &=\int_{-\infty}^{\infty}x\times\lambda e^{-\lambda x}dx \\
      &=\lambda\int_{-\infty}^{\infty}xe^{-\lambda x}dx \\
\end{align*}
We now make use of integration by parts with $f=x$ and $g'=e^{-\lambda x}$
\[
    \int xe^{-\lambda x}=-\frac{xe^{-\lambda x}}{\lambda}-\int-\frac{e^{-\lambda x}}{\lambda}dx
\]
And solve the integral using substitution with $u=-\lambda x\implies dx=\frac{1}{-\lambda}du$
\begin{align*}
    \int-\frac{e^{u}}{\lambda}\frac{1}{-\lambda}du\right)&=\frac{1}{\lambda^{2}}\int e^{u}du \\
      &=\frac{e^{u}}{\lambda^{2}} \\
      &=\frac{e^{-\lambda x}}{\lambda^{2}}
\end{align*}
As such we can compute the expected value as
\begin{align*}
    EX&=\int_{-\infty}^{\infty}\lambda\times\left(-\frac{xe^{-\lambda x}}{\lambda}-\frac{e^{-\lambda x}}{\lambda^{2}}\right)dx \\
      &=\int_{-\infty}^{\infty}-xe^{-\lambda x}-\frac{e^{-\lambda x}}{\lambda}dx \\
      &=\int_{-\infty}^{\infty}-\frac{\lambda xe^{-\lambda x}-e^{-\lambda x}}{\lambda}dx \\
      &=\left[\frac{(\lambda x+1)e^{-\lambda x}}{\lambda}\right]_{0}^{\infty} \\
      &=\frac{(\lambda\times 0+1)e^{0}}{\lambda}-\lim_{x\rightarrow \infty}\frac{(\lambda x+1)e^{-\lambda x}}{\lambda} \\
      &=\frac{1}{\lambda}
\end{align*}
To determine the variance we need $EX^{2}$
\begin{align*}
    EX^{2}&=\int_{0}^{\infty}x^{2}\times \lambda e^{-\lambda x}dx \\
       &=\lambda\int_{0}^{\infty}x^{2}\times e^{-\lambda x}dx \\
       &=\lambda\int_{0}^{\infty}\left(-\frac{x^{2}e^{-\lambda x}}{\lambda}-\int-\frac{2xe^{-\lambda x}}{\lambda}\right) \\
       &=\lambda\int_{0}^{\infty}\left(-\frac{x^{2}e^{-\lambda x}}{\lambda}+\frac{2}{\lambda}\int xe^{-\lambda x}\right) \\
       &=\lambda\int_{0}^{\infty}\left(-\frac{x^{2}e^{-\lambda x}}{\lambda}+\frac{2}{\lambda}\times\left(\frac{e^{-\lambda x}}{\lambda^{2}}\right)\right) \\
       &=\int_{0}^{\infty}-x^{2}e^{-\lambda x}+\frac{2e^{-\lambda x}}{\lambda^{2}} \\
       &=\int_{0}^{\infty}\frac{-\lambda^{2}x^{2}e^{-\lambda x}+2e^{-\lambda x}}{\lambda^{2}} \\
       &=\left[\frac{-\lambda^{2}x^{2}e^{-\lambda x}+2e^{-\lambda x}}{\lambda^{2}}\right]_{0}^{\infty} \\
       &=\frac{-\lambda^{2}0^{2}e^{-\lambda 0}+2e^{0}}{\lambda^{2}}-\lim_{x\rightarrow \infty}\frac{-\lambda^{2}x^{2}e^{-\lambda x}+2e^{-\lambda x}}{\lambda^{2}} \\
       &=\frac{2}{\lambda^{2}}
\end{align*}
As such we can determine the variance as
\begin{align*}
    \text{Var}(X)&=EX^{2}-(EX)^{2} \\
             &=\frac{2}{\lambda^{2}}-\frac{1}{\lambda^{2}} \\
             &=\frac{1}{\lambda^{2}}
\end{align*}
\subsubsection{Normal (Gaussian) distribution}
A standard normal random variable is a random variable whos PDF follows
\[
    f_{Z}(z)=\frac{1}{\sqrt{2\pi}}e^{-\frac{z^{2}}{2}}
\]
Where the first term ensures that the area under the curve is equal to one. The expected value of such a distribution is given by
\begin{align*}
    EZ&=\int_{-\infty}^{\infty}zf_{Z}(z)dz \\
      &=\int_{-\infty}^{\infty}z\times\frac{1}{\sqrt{2\pi}}e^{-\frac{z^{2}}{2}}dz \\
      &=\frac{1}{\sqrt{2\pi}}\int_{-\infty}^{\infty}z\times e^{-\frac{z^{2}}{2}}dz 
\end{align*}
Making use of integration by substitution with $u=-\frac{z^{2}}{2}\implies dz=\frac{1}{-z}du$
\begin{align*}
    EZ&=\frac{1}{\sqrt{2\pi}}\int_{-\infty}^{\infty}z\times e^{u}\times\frac{1}{-z}du \\
      &=-\frac{1}{\sqrt{2\pi}}\int_{-\infty}^{\infty}e^{u}du \\
      &=-\frac{1}{\sqrt{2\pi}}\left[e^{u}\right]_{-\infty}^{\infty} \\
      &=-\frac{1}{\sqrt{2\pi}}\left[e^{-\frac{z^{2}}{2}}\right]_{-\infty}^{\infty} \\
      &=-\frac{1}{\sqrt{2\pi}}\times\left(\lim_{z\rightarrow\infty}e^{-\frac{z^{2}}{2}}-\lim_{z\rightarrow-\infty}e^{-\frac{z^{2}}{2}}\right) \\
      &=0
\end{align*}
This makes sense as the distribution is symmetrical around 0. We now determine $EZ^{2}$ to enable determining the variance.
\begin{align*}
    EZ^{2}&=\int_{-\infty}^{\infty}z^{2}f_{Z}(z)dz \\
       &=\frac{1}{\sqrt{2\pi}}\int_{-\infty}^{\infty}z^{2}e^{-\frac{z^{2}}{2}}dz
\end{align*}
Let $u=-\frac{z^{2}}{2}\implies dz=-\frac{1}{-z}du$, by substitution we get
\begin{align*}
    EZ^{2}&=\frac{1}{\sqrt{2\pi}}\int_{-\infty}^{\infty}z^{2}e^{u}\frac{1}{-z}du \\
          &=\frac{1}{\sqrt{2\pi}}\int_{-\infty}^{\infty}-ze^{u}du \\
          &=-\frac{1}{\sqrt{2\pi}}\int_{-\infty}^{\infty}ze^{u}du
\end{align*}
Using integration by parts with $f=z,g'=e^{u}$ we get that
\begin{align*}
    \int ze^{u}du&=ze^{u}-\int e^{u}du \\
           &=
\end{align*}
