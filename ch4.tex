\section{Chapter 4}
\subsection{Continuous random variables and their distributions}
Suppose we have an interval on the real line, $[a,b]$ with a uniformly distributed probability of a given value being picked, we know that the probability of each point $P(X=x)=0$ as there is an infinite amount of points, as such, it only makes sense to look at subintervals of the interval when it comes to probability. 

Simply from the definition of a CDF we have that $F_{X}(x<a)=0$, whilst $F_{X}(x\geq b)=1$, simultaneously we can establish a general equation for the probability of an interval as the proportion it constitutes of the total interval
\begin{align*}
    F_{X}(a\leq x_{1}\leq x_{2}\leq b)&=P(X\in[x_{1},x_{2}]) \\
                       &=\frac{x_{2}-x_{1}}{b-a}
\end{align*}
Using this definition we can create a CDF as
\[
    F_{X}(x)\begin{cases}0 & x<a \\ \frac{x-a}{b-a} & a\leq x\leq b \\ 1 & x\geq b\end{cases}
    
\]
Whether we use $<$ or $\leq$ (or the reverse), doesn't matter as the probability of each individual point is equal to 0, as such $P(X<2)=P(X\leq 2)$.
\begin{definition}
  A random variable $X$ with CDF $F_{X}(x)$ is said to be continuous if $F_{X}(x)$ is a continuous function for all $x\in \mathbb{R}$.
\end{definition}
\subsection{Probability density function}
As its impossible to define a PMF for a continuous function (as $P(X=x)=0$ for all $x\in \mathbb{R}$), as instead define the probability density function as 
\[
    f_{X}(x)=\lim_{\Delta\rightarrow0^{+}}\frac{P(x<X\leq x+\Delta)}{\Delta}
\]
We recall that $P(a<X\leq b)=F_{X}(b)-F_{X}(a)$, as such we rewrite the limit as
\[
    f_{X}(x)=\lim_{\Delta\rightarrow0^{+}}\frac{F_{X}(x+\Delta)-F_{X}(x)}{\Delta}
\]
We then recognize this as the definition of the derivate, as such we can write that
\[
    f_{X}(x)&=\lim_{\Delta\rightarrow0^{+}}\frac{F_{X}(x+\Delta)-F_{X}(x)}{\Delta}=F'_{X}(x)
\]
\begin{definition}
  Consider a continuous random variable $X$ with an absolutely continuous CDF $F_{X}(x)$. The function $f_{X}(x)$ defined by
  \[
  f_{X}(x)=\frac{d}{dx}F_{X}(x)=F'_{X}(x)$ if $F_{X}(x) 
  \]
  is differentiable at $x$ is called the probability density function of $X$.
\end{definition}
Using the general example from the previous section we determine the PDF as
\begin{align*}
    F'_{X}(x)&=\frac{d}{dx}\frac{x-a}{b-a} \\
             &=\frac{1}{b-a}\left(\frac{d}{dx}x+\frac{d}{dx}-a\right) \\
             &=\frac{1}{b-a}
\end{align*}
As such we can define the probability density function as
\[
    f_{X}(X)=\begin{cases}\frac{1}{b-a} & a<x<b \\ 0 & x\notin [a,b]\end{cases}
\]
As the PDF is the derivative of the CDF, we can integrate a segment of the interval and get the probability thereof assuming it is absolutely continuous in that interval
\[
    F_{X}(x)=\int_{-\infty}^{x}f_{X}(x)dx
\]
At the same time the integral over the entire real line must be equal to $1$ in accordance with the axioms of probability
\[
    \int_{-\infty}^{\infty}f_{X}(x)dx=1
\]
Whilst we can determine the probability of an interval as
\[
    P(a<X\leq b)=F_{X}(b)-F_{X}(a)=\int_{a}^{b}f_{X}(x)dx
\]
Similarly we can define the range of a random variable $X$ as the possible values of the random varibles where the PDF is larger than $0$, as such
\[
    R_{X}=\{x|f_{X}(x)>0\}
\]
\subsection{Expected value and variance}
