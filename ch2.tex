\section{Chapter 2}
\subsection{Counting}
For a finite sample space with equal probabilities we recall that
\[
    P(A)=\frac{|A|}{|S|}
\]
As such determining the probability is a counting problem, determining the cardinality of $A$ and $S$.
\begin{definition}[Multiplication principle]
    Suppose that we perform $r$ experiments such that the $k$'th experiment has $n_{k}$ possible outcomes for $k=\{1,2,\ldots,r\}$. Then there are a total of $n_{1}\times n_{2}\times\ldots\times n_{r}$ possible outcomes for the sequence of $r$ experiments.
\end{definition}
For counting problems, some general terminology is relevant:
\begin{itemize}
    \item[-] Sampling: Choosing a random element from a set.
    \item[-] With replacement, the sampled element is returned to the set and can therefore be drawn multiple times with repeated sampling.
    \item[-] Without replacement, the sampled element is not returned to the set and can therefore not be drawn multiple times.
    \item[-] Ordered means that the order at which elements are written matters, $\{a_{1},a_{2},a_{3}\}\neq \{a_{3},a_{1},a_{2}\}$.
    \item[-] Unordered means that the order at which elements are written does not matter, $\{a_{1},a_{2},a_{3}\}=\{a_{3},a_{1},a_{2}\}$.
\end{itemize}
\subsubsection{Ordered sampling with replacement}
Suppose we have a set consisting of $n$ elements and we wish to draw $k$ samples from the set, for example, say $A=\{1,2,3\}$ where we wish to sample $k=2$, we then get 9 different possibilities
\[
    (1,1),(1,2),(1,3),(2,1),(2,2),(2,3),(3,1),(3,2),(3,3)
\]
From this its clear that we create a list consisting of $k$-valued elements where each position has $n$ options for values
\[\begin{array}{cccc}
    a_1 & a_2 & \ldots & a_k \\
    \uparrow & \uparrow & & \uparrow \\
    n & n & & n
 \end{array}\]
Meaning that we can determine the total amount of possibilities as 
\[
    n\times n\times\ldots\times n=n^{k}
\]
\subsubsection{Ordered sampling without replacement (Permutations)}
As opposed to the previous circumstance, a element is now removed every time we draw, resulting in there being one less options every time we move to the next position, as such, using the same example with $A$ and $k=2$ we get
\[
    (1,2),(1,3),(2,1),(2,3),(3,1),(3,2)
\]
Here we create a list of $k$-valued elements where each position has one less option than the previous
\[\begin{array}{cccc}
    (a_1 & a_2 & \ldots & a_k) \\
    \uparrow & \uparrow & & \uparrow \\
    n & n-1 & & n-k+1
 \end{array}\]
Which is called a $k$-permutation of the elements in the set, the following notation is used to show the number of $k$-permutations of an $n$-element set
\[
    P_{k}^{n}=n\times(n-1)\times\ldots\times(n-k+1)
\]
A special case can also occur here when $n<k$, as there then wont be enough options for every position and there therefore are no possible lists. Another special case is an $n$-permutation where $k=n$ which results in the sequence
\begin{align*}
    P_{n}^{n}&=n\times(n-1)\times(n-2)\times\ldots\times(n-n+1) \\
         &=n\times(n-1)\times(n-2)\times\ldots\times 1 \\
         &=n!
\end{align*}
As such the factorial operator simply denotes the total number of permutations of an $n$ element set, aka the total number of ways you can order $n$ different objects. By definition $0!=1$, using this we rewrite the formula for $P_{k}^{n}$
\begin{theorem}
  The amount of $k$-permutations of an $n$-element set is given by $P_{k}^{n}=\frac{n!}{(n-k)!}$.
\end{theorem}
\begin{proof}
  From the original expression for $P_{k}^{n}$ we have that
  \[
      P_{k}^{n}&=n\times(n-1)\times(n-2)\times\ldots\times(n-k+1) \\
  \]
  By multiplying by $\frac{(n-k)!}{(n-k)!}$ we get
\begin{align*}
         &=n\times(n-1)\times(n-2)\times\ldots\times(n-k+1)\cdot\frac{(n-k)!}{(n-k)!} \\
         &=\frac{n!}{(n-k)!}
\end{align*}
  As multiplying by $(n-k)!$ in the numerator results in the sequence ``finishing'' and being equal to $n!$.
\end{proof}
\subsubsection{Unordered sampling without replacement (Combinations)}
We now wish to determine the amount of possible lists when we sample $k$ elements from an $n$-element set. This means that we want to determine the amount of possible $k$-element subsets of the $n$-element set. Using the same example as before with $A$ and $k=2$ we get 3 different combinations
\[
    (1,2),(1,3),(2,3)
\]
We show the number of $k$-element subsets of $A$ as
\[\begin{pmatrix}
  n \\
  k
 \end{pmatrix}\]
Which is read as ``$n$ choose $k$'', to determine the value of this we compare with $P_{k}^{n}$ as the only difference between the two is ordering. This is because for any $k$-element subset of an $n$-element set, we can order the elements in $k!$ different ways, as such
\[
    P_{k}^{n}=\begin{pmatrix}n\\k\end{pmatrix}\times k!
\]
Rewriting using our previously established formula for $P_{k}^{n}$ and dividing by $k!$ we get
\[
    \begin{pmatrix}n\\k\end{pmatrix}=\frac{n!}{k!(n-k)}\text{, for }0\leq k\leq n
\]
This term is used extensively in the binomial theorem, which states that
\[
    (a+b)^{n}=\sum_{k=0}^{n}\begin{pmatrix}n\\k\end{pmatrix}a^{k}b^{n-k}
\]
\begin{theorem}
    For any non-negative integers $n,k$ it follows that $\begin{pmatrix}n\\k\end{pmatrix}=\begin{pmatrix}n\\n-k\end{pmatrix}$.
\end{theorem}
\begin{proof}
  Assume we wish to determine the amount of possible sequences consisting of $k$ A's and $j$ B's, as such we have $n=j+k$ positions to fill with either A or B, from these positions we need to choose $j$ for A's and whatever is left is filled with B's, as such the amount of ways is
  \[
      \begin{pmatrix}n\\j\end{pmatrix}
  \]
  If we instead observe this from the point of B's, it is clear that the amount of ways would then be given by
  \[
      \begin{pmatrix}n\\k\end{pmatrix}
  \]
  As these must be equivalent we have that
  \[
      \begin{pmatrix}n\\j\end{pmatrix}=\begin{pmatrix}n\\k\end{pmatrix}
  \]
  From the initial determination of $n$ we get that
  \[
      n=j+k\implies j=n-k\vee k=n-j
  \]
  As such
  \[
      \begin{pmatrix}n\\j\end{pmatrix}=\begin{pmatrix}n\\n-k\end{pmatrix}=\begin{pmatrix}n\\k\end{pmatrix}=\begin{pmatrix}n\\n-j\end{pmatrix}
  \]
\end{proof}
\begin{theorem}
    For any non-negative integers $k,n$ it follows that $\sum_{k=0}^{n}\begin{pmatrix}n\\k\end{pmatrix}=2^{n}$.
\end{theorem}
\begin{proof}
  From the Binomial theorem we know that
  \[
      (a+b)^{n}=\sum_{k=0}^{n}\begin{pmatrix}n\\k\end{pmatrix}a^{k}b^{n-k}
  \]
  We let $a=b=1$ and as such get
  \[
      2^{n}=\sum_{k=0}^{n}\begin{pmatrix}n\\k\end{pmatrix}
  \]
\end{proof}
\begin{theorem}
    For non-negative integers $0\leq k\leq n$ it follows that $\begin{pmatrix}n+1\\k+1\end{pmatrix}=\begin{pmatrix}n\\k+1\end{pmatrix}+\begin{pmatrix}n\\k\end{pmatrix}$.
\end{theorem}
\begin{proof}
  We define an arbitrary set, $A$ with $n+1$ elements
  \[
      A=\{a_{1},a_{2},\ldots,a_{n},a_{n+1}\}
  \]
  From this set we wish to choose a $k+1$ element subset, call it $B$, by combinations we know this is equal to
  \[
      \begin{pmatrix}n+1\\k+1\end{pmatrix}
  \]
  $B$ can also be constructed as the union of two subsets of $B$ that are defined by either containing- or not containing $a_{n+1}$
  \[
      B=B_{1}\cup B_{2}\text{, where } a_{n+1}\notin B_{1},a_{n+1}\in B_{2},B_{1}\cap B_{2}=\emptyset
  \]
  To define $B_{1}$ we need to choose $k+1$ elements from the set $A\setminus a_{n+1}$ which is equal to
  \[
      \begin{pmatrix}n\\k+1\end{pmatrix}
  \]
  To complete the set we then need to choose $k$ elements from $A$ which can be done in
  \[
      \begin{pmatrix}n+1\\k\end{pmatrix}
  \]
  ways. As such the sum of the two must be equal to the initial expression resulting in
  \[
      \begin{pmatrix}n+1\\k+1\end{pmatrix}=\begin{pmatrix}n\\k+1\end{pmatrix}+\begin{pmatrix}n\\k+1\end{pmatrix}
  \]
\end{proof}
\begin{theorem}
    Vandermonde's identity states that $\begin{pmatrix}m+n\\k\end{pmatrix}=\sum_{i=0}^{k}\begin{pmatrix}m\\i\end{pmatrix}\begin{pmatrix}n\\k-i\end{pmatrix}$
\end{theorem}
\begin{proof}
  We construct a set $A$ with $m+n$ elements, as such
  \[
      A=\{a_{1},a_{2},\ldots,a_{m},b_{1},b_{2},\ldots,b_{n}\}
  \]
  Determining the number of $k$-element subsets of $A$ is equal to
  \[
      \begin{pmatrix}m+n\\k\end{pmatrix}
  \]
  This can also be done by choosing $i$ elements from $\{a_{1},a_{2},\ldots,a_{n}$ first, and then $k-i$ elements from $\{b_{1},b_{2},\ldots,b_{n}\}$, which then can be done in
  \[
      \begin{pmatrix}m\\i\end{pmatrix}\begin{pmatrix}n\\k-i\end{pmatrix}
  \]
  ways. But as $i$ can be any number from $0\rightarrow k$ it is necessary to sum all the possible options whereby we write
  \[
      \begin{pmatrix}m+n\\k\end{pmatrix}=\sum_{i=0}^{k}\begin{pmatrix}m\\i\end{pmatrix}\begin{pmatrix}n\\k-i\end{pmatrix}
  \]
\end{proof}
An important class of random experiments are Bernoulli trials, a random experiemnt where there are two possible outcomes, succes and failure (which can be extended to any experiment with 2 outcomes as we can arbitrarily define success and failure).

In Bernoulli trials the probability of success if usually denoted by $p$ and the probability of failure as its complement $q=1-p$. If we perform $n$ independent Bernoulli trials and count the number of successes, it is called a binomial experiment, for example a coin toss where we define success as heads and count the number of heads. 
\begin{theorem}
    The binomial formula is given by $P(k)=\begin{pmatrix}n\\k\end{pmatrix}p^{k}(1-p)^{n-k}$
\end{theorem}
\begin{proof}
    Imagine we toss a coin with $P(H)=p$ and $P(T)=1-p$ $n$ times, we define $C$ as the event of observing $k$ heads (and $n-k$ tails), the probability of observing $k$ heads will be given by
    \[
        P(k)=|C|p^{k}(1-p)^{n-k}
    \]
    To determine $|C|$, we realise that we can see the event as ordered sampling without replacement, as such we have that
    \[
        |C|=\begin{pmatrix}n\\k\end{pmatrix}
    \]
    Which we insert into the previous expression and get
    \[
        P(k)=\begin{pmatrix}n\\k\end{pmatrix}p^{k}(1-p)^{n-k}
    \]
\end{proof}
\subsubsection{Unordered sampling without replacement}
As opposed to the previous section we're now working with replacement, as such we again use the example of $A$ with $k=2$ and get 6 possibilities given by
\[
    (1,1),(1,2),(1,3),(2,2),(2,3),(3,3)
\]
One way to represent this sample is to list them as $n$-element vectors where each position corresponds to a number, eg.
\begin{align*}
    (a,b)&\rightarrow(x_{1},x_{2},x_{3})=(n_{1},n_{2},n_{3}) \\
    (1,1)&\rightarrow(x_{1},x_{2},x_{3})=(2,0,0) \\
    (1,2)&\rightarrow(x_{1},x_{2},x_{3})=(1,1,0) \\
    (1,3)&\rightarrow(x_{1},x_{2},x_{3})=(1,0,1) \\
\end{align*}
Constructing these vectors a pattern emerges, that is $\sum_{i=1}^{3}x_{i}=2$, as such we can determine the amount of possibilities as the amount of integer solutions to $x_{1}+x_{2}+x_{3}=2$.
\begin{theorem}
    The number of distinct solutions to the equation $x_{1}+x_{2}+\ldots+x_{n}=k$ where $x_{n}\in\{0,1,2,3,\ldots\}$ is equal to $\begin{pmatrix}n+k-1\\k\end{pmatrix}=\begin{pmatrix}n+k-1\\n-1\end{pmatrix}$.
\end{theorem}
\begin{proof}
  We first define a mapping where an integer $x_{n}\geq 0$ is replaced with $x_{n}$ vertical lines. Suppose we now have a solution to the equation where $k=6$, for example
  \[
      3+0+2+1\Leftrightarrow |||++||+|
  \]
  Here we realise that the equation can be represented by $k$ vertical lines and $n-1$ plus signs, as such we can use combinations to determine the amount of distinct sequences we can create using $k$ vertical lines and $n-1$ plus signs, as such
  \[
      n_{\text{solutions}}=\begin{pmatrix}k+n-1\\k\end{pmatrix}=\begin{pmatrix}k+n-1\\n-1\end{pmatrix}
  \]
\end{proof}
\subsubsection{Problems}
\paragraph{Problem 1}
Let $A,B$ be two finite sets with $|A|=m,|B|=n$, how many distinct functions (mappings) can you define from set $A$ to set $B$, $A\xrightarrow{f} B$?

We let
\begin{align*}
    A&=\{a_{1},a_{2},\ldots,a_{m}\} \\
    B&=\{b_{1},b_{2},\ldots,b_{n}\}
\end{align*}
For each $a_{m}$ we have $n$ options as
\[
    f(a_{m})\in B
\]
As such we by the multiplication principle have that there exists
\[
    n^{m}
\]
distinct mappings.
\paragraph{Problem 2}
A function is said to be one-to-one if for all $x_{1}\neq x_{2}\implies f(x_{1})\neq f(x_{2})$. Equivalently we can say a function is one-to-one whenever $f(x_{1})=f(x_{2})\implies x_{1}=x_{2}$. Let $A,B$ be two finite sets with $|A|=m,|B|=n$, how many distinct one-to-one functions (mappings) can you define from set $A$ to set $B$, $A\xrightarrow{f} B$.

Again we define
\begin{align*}
    A&=\{a_{1},a_{2},\ldots,a_{m}\} \\
    B&=\{b_{1},b_{2},\ldots,b_{n}\}
\end{align*}
As opposed to the previous problem we here lose a possible value every time we progress once, this means that $a_{1}$ has $n$ options, $a_{2}$ has $n-1$ options, etc. as such we're making a permutation resulting in the amount of options that exists being given by
\[
    P_{m}^{n}=\frac{n!}{(n-m)!}
\]
\paragraph{Problem 3}
An urn contains 30 red balls and 70 green balls. What is the probability of getting exactly $k$ red balls in a sample of size $20$ if the sampling is done with replacement? Assume $0\leq k\leq 20$.

The probability of picking a red ball is 
\[
    P(R)=\frac{30}{100}=0.3
\]
And as such the chance of not drawing one is
\[
    P(G)=1-0.3=0.7
\]
As we're working with replacement, this stays constant throughout all samples, as such we can determine the probability of drawing $k$ balls using the binomial formula
\[
    P(k)=\begin{pmatrix}20\\k\end{pmatrix}0.3^{k}(0.7)^{20-k}
\]
\paragraph{Problem 4}
An urn contains 30 red balls and 70 green balls. What is the probability of getting exactly $k$ red balls in a sample of size $20$ if the sampling is done without replacement?

We let $A$ be the event of getting $k$ red balls, to determine $P(A)$ we need to find
\[
    P(A)=\frac{|A|}{|S|}
\]
Determining $|S|$ is trivial
\[
    |S|=\begin{pmatrix}100\\20\end{pmatrix}
\]
We now define $m$ as $|R|$ as $n$ as $|G|$, whilst $k=|R|$. As such we make use of
\[
    |A|=\begin{pmatrix}m\\k\end{pmatrix}\begin{pmatrix}n\\20-k\end{pmatrix}
\]
Inserting our known information we then get that
\[ 
    |A|=\begin{pmatrix}30\\k\end{pmatrix}\begin{pmatrix}70\\20-k\end{pmatrix}
\]
By using the initial expression we then get
\[
    P(A)=\frac{\begin{pmatrix}30\\k\end{pmatrix}\begin{pmatrix}70\\20-k\end{pmatrix}}{\begin{pmatrix}100\\20\end{pmatrix}}
\]
\paragraph{Problem 5}
Assume there are $k$ people in a room and we know that
\begin{itemize}
    \item[-] $P(k=5)=\frac{1}{4}$
    \item[-] $P(k=10)=\frac{1}{4}$
    \item[-] $P(k=15)=\frac{1}{2}$
\end{itemize}
a) What is the probability that at least two of them have been born in the same month? Assume that all months are equally likely.

We let $A$ be the event that two people are born in the same month, as we the phrase at least suggests determining the complement might be easier we wish to find
\[
    P(A)=1-\frac{|\overline{A}|}{|S|}
\]
Determining $|S|$ is trivial by the multiplication principle
\[
    |S|=12^{k}
\]
To determine the amount of possible we can use the same principle, except repetition isnt allowed, as such
\[
    |\overline{A}|=P_{k}^{12}
\]
Whereby it becomes clear that
\[
    P(A_{k})=1-\frac{P_{k}^{12}}{12^{k}}
\]
By the law of total probability we then have that
\begin{align*}
    P(A)&=P(A_{5})\cdot\frac{1}{4}+P(A_{10})\cdot\frac{1}{4}+P(A_{15})\frac{1}{2} \\
        &=\frac{1}{4}\left(1-\frac{P_{5}^{12}}{12^{5}}\right)+\frac{1}{4}\left(1-\frac{P_{10}^{12}}{12^{10}}\right)+\frac{1}{2}
\end{align*}
As $P(A_{15})$ will always be $=1$ since $k>12$.

b) Given that we already know there are at least two people that celebrate their birthday in the same month, what is the probability that $k=10$?

We are asked to determine the conditional probability
\[
    P(k=10|A)
\]
By Bayes law we have that
\begin{align*}
    P(k=10|A)&=\frac{P(A|k=10)P(k=10)}{P(A)} \\
             &=\frac{\left(1-\frac{P_{10}^{12}}{10^{12}}\right)\cdot\frac{1}{4}}{\frac{1}{4}\left(1-\frac{P_{5}^{12}}{12^{5}}\right)+\frac{1}{4}\left(1-\frac{P_{10}^{12}}{12^{10}}\right)+\frac{1}{2}} \\
             &=\frac{1-\frac{P_{10}^{12}}{10^{12}}}{\left(1-\frac{P_{5}^{12}}{12^{5}}\right)+\left(1-\frac{P_{10}^{12}}{12^{10}}\right)+2}
\end{align*}
\paragraph{Problem 6}
How many distinct solutions does the following equation have?
\begin{equation*}
    \begin{gathered}
        x_{1}+x_{2}+_{3}+x_{4}=100\text{, such that } x_{1}\in\{1,2,3,\ldots\}, \\
        x_{2}\in\{2,3,4,\ldots\},x_{3},x_{4}\in\{0,1,2,3,\ldots\}
    \end{gathered}
\end{equation*}
We know that the number of solutions without the given restrictions is given by
\[
    \begin{pmatrix}k+n-1\\k\end{pmatrix}=\begin{pmatrix}k+n-1\\n-1\end{pmatrix}
\]
We wish to rewrite the given equation such that these restrictions are no longer present, as such we rewrite
\begin{align*}
    y_{1}&=x_{1}-1 \\
    y_{2}&=x_{2}-2
\end{align*}
As this gives them the same domain, the equation is now
\[
    y_{1}+1+y_{2}+2+x_{3}+x_{4}=100\text{, where }y_{1},y_{2},x_{3},x_{4}\in\{0,1,2,3,\ldots\}
\]
Rewriting again this gives
\[
    y_{1}+y_{2}+x_{3}+x_{4}=97
\]
Which has the solution
\[
    \begin{pmatrix}97+4-1\\97\end{pmatrix}=\begin{pmatrix}100\\3\end{pmatrix}
\]
\paragraph{Problem 7}
$N$ guests arrive at a party. Each person is wearing a hat. We collect all hats and then randomly redistribute the hats, giving each person one of the $N$ hats randomly.What is the probability that at least one person receives his/her own hat?
\subsection{Chapter resume}
  Ordered sampling with replacement
  \[
      n^{k}
  \]
  Ordered sampling without replacement
  \[
      P_{k}^{n}=\frac{n!}{(n-k)!}
  \]
  Unordered sampling without replacement
  \[
      \begin{pmatrix}n\\k\end{pmatrix}=\frac{n!}{k!(n-k)!}
  \]
  Unorderered sampling with replacement
  \[
      \begin{pmatrix}k+n-1\\k\end{pmatrix}=\begin{pmatrix}k+n-1\\n-1\end{pmatrix}
  \]
