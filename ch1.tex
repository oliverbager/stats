\section{Chapter 1}
\subsection{Set operations}
A union of 2 sets is given by the combination of their elements:
\[
    A\cup B = \{1,2\}\cup\{2,3\}=\{1,2,3\}
\]
The intersection of 2 sets is instead given by their shared elements:
\[
    A\cap B=\{1,2\}\cap\{2,3\}=\{2\}
\]
\begin{theorem}[De Morgan's law]
  For any sets $A_1,A_2,\ldots,A_n$ we have
  \begin{align*}
    \overline{A_1\cup A_2\cup \ldots \cup A_n}&=\overline{A_1}\cap\overline{A_2}\cap\ldots\cap\overline{A_n} \\
    \overline{A_1\cap A_2\cap\ldots\cap A_n}&=\overline{A_1}\cup\overline{A_2}\cup\ldots\cup\overline{A_n}
  \end{align*}
\end{theorem}

\begin{theorem}[Distributive law]
  For any sets $A$, $B$ and $C$ we have
  \begin{align*}
      A\cap(B\cup C)&=(A\cap B)\cup (A\cap C) \\
      A\cup(B\cap C)&=(A\cup B)\cap (A\cup C)
  \end{align*}
\end{theorem}
The complement of a set is given by all elements that are in the universal set, but not the set itself:
\[
    S=\{1,2,3,4,5\}\hskip 32pt \overline{A}=S\setminus A=\{1,2,3,4,5\}\setminus\{1,2\}=\{3,4,5\}
\]
The difference between two sets is given by elements in the first but not the second:
\[
    A\setminus B=\{1,2\}-\{2,3\}=\{1\}\hskip 32pt A\setminus B=A\cap\overline{B}
\]
Two sets are disjoint if their intersection is an empty set
\[
    A\cap B=\emptyset
\]
Sets can be partitioned into smaller parts. The sets $A_1,A_2,\ldots,A_n$ are a partition of $S$ if they're disjoint and:
\[
    \bigcup_{i=1}^n A_i=S
\]
The cartesian product of two sets are given by the ordered pairs of both sets:
\[
    A\times B=\{1,2\}\times\{2,3\}=\{(1,2),(1,3),(2,2),(2,3)\}
\]
Which can be expressed more generally as:
\[
    A\times B=\{(x,y)|x\in A \text{ and } y\in B\}
\]
The number of elements contained in a (finite) sets is given by its cardinality:
\[
    |A|=|\{1,2\}|=2
\]
For determining the cardinality of (finite) sets, the inclusion-exclusion principle is often used:
\[
    |A\cup B|=|A|+|B|-|A\cap B|
\]
This can once again be expanded to more sets:
\begin{equation*}
    \begin{gathered}
    |\bigcup_{i=1}^n A_i|=\sum_{i=1}^n |A_i|-\sum_{i<j}|A_i\cap A_j| \\
    +\sum_{i<j<k}|A_i\cap A_j\cap A_k|-\cdots +(-1)^{n+1}|A_1\cap \cdots\cap A_n|
    \end{gathered}
\end{equation*}
\subsubsection{Cardinality and countable sets}
Finite sets are obviously countable, however when we move onto infinite sets they are divided into countable \textbf{and} uncountable sets. A countable set is characterised by the ability to write it in one-to-one correspondance with the natural numbers, e.g.:
\[
    A=\{a_1,a_2,\ldots,a_n\}
\]
Meaning you can list the elements, this is true for sets like the natural numbers, $\mathbb{N}$, and the integers, $\mathbb{Z}$, but also the rational numbers, $\mathbb{Q}$. Uncountable sets (such as the real- and complex numbers) on the other hand cannot be written as lists, but instead have to be denoted as intervals.
\begin{definition}[Countability of a set]
  A set, $A$, is called countable if one of the following is true:
  \begin{itemize}
      \item[-] It is a finite set, $|A|<\infty$.
      \item[-] The set can be written as a list with one-to-one correspondance with the natural numbers.
  \end{itemize}
\end{definition}
This means that any subset of $\mathbb{N}, \mathbb{Z} \text{ and } \mathbb{Q}$ are countable, whilst any set containing an interval on the real line is uncountable. 
\begin{theorem}[Countability of sub- and supersets]
  Any subset of a countable set is countable and any superset of an uncountable set is uncountable.
\end{theorem}
\begin{proof}
  Let $A$ be a countable set and $B\subset A$. If $A$ is finite, then it follows that $|B|\leq|A|<\infty$, thus $B$ must be countable as its cardinality cannot exceed that of $A$, which must be smaller than $\infty$.

  If $A$ is instead countably infinite, then it follows that as $B$ is a subset of $A$ it must be possible to construct it by removing $\overline{B}$ from $A$, whereby it must also be countable, as it can be constructed as a list. 

  The opposite can be argued by assuming $B$ is \textbf{not} countable, whereby a contradiction would occur in both proofs.
\end{proof}
\begin{theorem}[Countability of union]
  If $A_1,A_2,\ldots,A_n$ are countable sets, then the union of those must also be countable.
\end{theorem}
\begin{proof}
  As the sets are countable it must be possible to write them in the form
  \begin{align*}
      A_1&=\{a_{11},a_{12},\ldots,a_{1n}\} \\
      A_2&=\{a_{21},a_{22},\ldots,a_{2n}\} \\
      A_3&=\{a_{31},a_{32},\ldots,a_{3n}\}
  \end{align*}
  As such the union of those sets must also be possible to construct as a list
  \[
      \bigcup_{i=1}^{m}A_{i}=\{a_{11},a_{12},a_{21},a_{22},a_{31},a_{32},\ldots,a_{mn}\}
  \]
  And as a result must be countable.
\end{proof}
\begin{theorem}[Countability of carthesian product]
  If $A$ and $B$ are countable, then $A\times B$ is also countable.
\end{theorem}
\begin{proof}
  As $A$ and $B$ are countable it must be possible to write them in the form
  \begin{align*}
      A&=\{a_{1},a_{2},\ldots,a_{n}\} \\
      B&=\{b_{1},b_{2},\ldots,b_{n}\}
  \end{align*}
  In accordance with the definition of the carthesian product, the two sets can be constructed as a list with the form
  \[
      A\times B=\{(a_{i},b_{j})|i,j\in \mathbb{N}\}
  \]
  Whereby it must be countable as it can be constructed as a list.
\end{proof}
As a result of this proof it also becomes clear that any set that can be written in the form
\[
    C=\bigcup_{i}\bigcup_{j}\{a_{ij}\}\text{ where }i,j \text{ belong to a countable set}
\]
Must also be countable, the set of rational numbers is an example of this as it can be written as
\[
    \mathbb{Q}=\bigcup_{i\in \mathbb{Z}}\bigcup_{j\in \mathbb{N}}\left\{\frac{i}{j}\right\}
\]
\subsubsection{Functions}
Functions take an input from its domain, apply a rule to said input, whereby an output from the co-domain is produced.
\[
    f:A\xrightarrow{} B\hskip 32pt f(x\in A)\in B
\]
\begin{definition}
  A function maps elements from the domain set to elements in the co-domain with the property that each input is mapped to exactly one output.
\end{definition}
In the same context the range operand is defined, as it is not necessary for a function to be able to output all elements of the codomain:
\[
    f: \mathbb{R} \xrightarrow{x^{2}} \mathbb{R}
\]
Here both the domain- and co-domain are the real numbers, however it is clear that no value $x\in \mathbb{R}$ would ever produce a negative number, therefore:
\[
    \text{Range}(f)=\mathbb{R}^{+}
\]
\subsection{Random experiments}
A random experiment will always have an \textbf{outcome} corresponding to an element from the \textbf{sample space}, $S$.
\begin{definition}
  A random experiment is a process by which we observe something uncertain.
\end{definition}
When a random experiment is repeated, each repetition is called a \textbf{trial}. The goal of analyzing a random experiment is to assign probabilities to \textbf{events}, which correspond to subsets of the sample space.

\subsubsection{Probability}
A probability is assigned to an event, $P(A)\in[0,1]$. The derivation of probability theorem is based on 3 axioms:
\begin{itemize}
    \item Axiom 1: For any event $A$, $1\geq P(A)\geq 0$.
    \item Axiom 2: Probability of the sample space, $S$, is $P(S)=1$.
    \item Axiom 3: If $A_{1},A_{2},\ldots,A_{n}$ are disjoint events, then $P(\bigcup_{i=1}^{n}A_{i})=\sum_{i=1}^{n}P(A_{i})$
\end{itemize}
Notationally unions and intersections can be read as:
\begin{align*}
    P(A\cap B)&=P(A \text{ and } B)=P(A,B) \\
    P(A\cup B)&=P(A \text{ or } B)
\end{align*}
\begin{theorem}[Probability of complement]
  For any event $A$, $P(\overline{A})=1-P(A)$.
\end{theorem}
\begin{proof}
  As the complement of a set contains all elements of the sample space that are not in the set
  \[
      \overline{A}=S\setminus A
  \]
  It is clear that their unions must be $S$ and they must be disjoint whereby
  \[
      P(A\cup \overline{A})=P(S)=1
  \]
  As they are disjoint we can write the probability of their union as the sum of their probabilities
  \[
      P(A)+P(\overline{A})=1\Leftrightarrow P(A)=1-P(\overline{A})
  \]
\end{proof}
\begin{theorem}[Probability of empty set]
  The probability of the empty is zero, $P(\emptyset)=0$.
\end{theorem}
\begin{proof}
  As the empty set must be the complement of the sample space we have that
  \[
      P(\emptyset)=P(\overline{S})=1-P(S)=1-1=0
  \]
\end{proof}
\begin{theorem}[Probability must be equal to or less than 1]
  For any event $A$, $P(A)\leq 1$.
\end{theorem}
\begin{proof}
  By the first axiom we have that
  \[
      P(\overline{A})\geq 0
  \]
  It becomes clear that
  \[
      P(A)\leq 1
  \]
  As $P(A)+P(\overline{A})=1$.
\end{proof}
\begin{theorem}[Probability of a difference]
  The probability of a difference is given by $P(A\setminus B)=P(A)-P(A\cap B)$.
\end{theorem}
\begin{proof}
  As $A\cap B$ and $A\setminus B$ must be disjoint, whilst their union must be $A$
  \[
      (A\cap B)\cup(A\setminus B)=A
  \]
  We have by the third axiom that
  \[
      P(A)=P((A\cap B)\cup(A\setminus B))=P(A\cap B)+P(A\setminus B)
  \]
  By rearranging it becomes clear that
  \[
      P(A\setminus B)=P(A)-P(A\cap B)
  \]
\end{proof}
\begin{theorem}[Probability of a union]
  The probability of a union is given by $P(A\cup B)=P(A)+P(B)-P(A\cap B)$.
\end{theorem}
\begin{proof}
  As $A$ and $B\setminus A$ must be disjoint sets whilst their union must be $A\cup B$, it is clear that
  \[
      P(A\cup B)=P(A\cup (B\setminus A))
  \]
  As we know these are disjoint we write
  \[
      P(A\cup B)=P(A)+P(B\setminus A)
  \]
  Rewriting using the previous theorem we then have
  \[
      P(A\cup B)=P(A)+P(B)-P(A\cap B)
  \]
\end{proof}
\begin{theorem}[Probability of a subset must be less than or equal to its superset]
  If $A\subset B$ then $P(A)\leq P(B)$.
\end{theorem}
\begin{proof}
  As $A\subset B$ it is clear that their union must be $B$
  \[
      P(B)=P(A\cap B)+P(B\setminus A)
  \]
  As their intersection is $A$ we have that
  \[
      P(B)=P(A)+P(B\setminus A)
  \]
  As
  \[
      P(B\setminus A)\geq 0
  \]
  By the first axiom, we have that
  \[
      P(B)\geq P(A)
  \]
\end{proof}
\subsection{Conditional probability}
Conditional probabilities are written as
\[
    P(A|B)
\]
And are read as ''\textit{the probability of $A$, given that $B$ has occurred}``. The conditional probability will therefore be given by
\begin{align*}
    P(A|B)&=\frac{|A\cap B|}{|B|} \\
            &=\frac{\frac{|A\cap B|}{|S|}}{\frac{|B|}{|S|}} \text{ dividing by $|S|$} \\
            &=\frac{P(A\cap B)}{P(B)}
\end{align*}
This is as when we know $B$ has occured, the sample space of $A$ is shrunk to $B$, whereby the cardinality of their intersection must be equal to the amount of favourable outcomes.

The earlier established probability axioms can also be formulated for conditional probabilities
\begin{itemize}
    \item[-] Axiom 1: For any event $A$, $P(A|B)\geq 0$.
    \item[-] Axiom 2: Conditional probability of $B$ given $B$ is $1$, i.e., $P(B|B)=1$.
    \item[-] Axiom 3: If $A_{1},A_{2},\ldots,A_{n}$ are disjoint events, then $P(\bigcup_{i=1}^{n}A_{i}|B)=\sum_{i=1}^{n} P(A_{i}|B)$.
\end{itemize}
And the same applies to the established fomulas

\begin{theorem}
  For any conditional event, $A|C$, $P(\overline{A}|C)=1-P(A|C)$.
\end{theorem}
\begin{proof}
  We know that
  \[
      P(A|B)=\frac{P(A|B)}{P(B)}
  \]
  Assuming the theorem is correct we then have that
  \[
      1-P(\overline{A}|C)=1-\frac{P(\overline{A}|C)}{P(B)}
  \]
  We wish to show that
  \[
      \frac{P(A|B)}{P(B)}=1-\frac{P(\overline{A}|C)}{P(B)}
  \]
  Multiplying by $P(B)$ gets us
  \[
      P(A|B)=P(B)-P(\overline{A}|C)
  \]
  Adding $P(\overline{A}|C)$ on the LHS we get
  \[
      P(A|B)+P(\overline{A}|C)=P(B)
  \]
  As the two are mutually exclusive by the definition of the complement
  \[
      P((A|B)\cup (\overline{A}|C))=P(B)
  \]
  Which is true.
\end{proof}
\begin{theorem}
  The probability of the empty set is zero, $P(\emptyset|C)=0$.
\end{theorem}
\begin{proof}
  As the empty is the complement of the sample set we have from the previous theorem that
  \[
      P(S|C)=1
  \]
  By applying previous equation
  \[
      P(\overline{S}|C)=1-1=0
  \]
\end{proof}
\begin{theorem}
  The probability of a conditional probability occuring must always be less than or equal to 1, $P(A|C)\leq 1$.
\end{theorem}
\begin{proof}
  From the definition of conditional probability
  \[
      P(A|B)=\frac{P(A\cap B)}{P(B)}
  \]
  It is clear that the denominator and numerator can never be more than even as $A\cap B=B$ even if $A\superset B$, resulting in the maximum value being 1.
  \[
      P(A|B)\leq 1
  \]
\end{proof}
\begin{theorem}
  The probability of a difference is given by $P(A\setminus B|C)=P(A|C)-P(A\cap B|C)$.
\end{theorem}
\begin{proof}
  As $A\cap B|C$ and $A\setminus B|C$ must be disjoint whilst their union must be $A|C$, we have that
  \[
      (A\cap B|C)\union (A\setminus B|C)=A|C
  \]
  By the third axiom we have that
  \[
      P(A|C)=P(A\cap B|C)+P(A\setminus B|C)
  \]
  Rearranging the terms we get
  \[
      P(A\setminus B|C)=P(A|C)-P(A\cap B|C)
  \]
\end{proof}
\begin{theorem}
  The probability of a union is given by $P(A\cup B|C)=P(A|C)+P(B|C)-P(A\cap B|C)$.
\end{theorem}
\begin{proof}
  As $A|C$ and $B\setminus A|C$ must be disjoint and their union must be equal to $A\cup B|C$ we have that
  \[
      P(A\cup B)=P(A|C\union (B\setminus A|C))
  \]
  As these sets are disjoint we rewrite using the third axiom
  \[
      P(A\cup B)=P(A|C)+P(B\setminus A|C)
  \]
  By the previous theorem the difference is rewritten as
  \[
      P(A\cup B)=P(A|C)+P(B|C)-P(A\cap B|C)
  \]
\end{proof}
\begin{theorem}
  If $A\subset B$ then $P(A|C)\leq P(B|C)$.
\end{theorem}
\begin{proof}
  As $A\subset B$ it is clear that their union must be $B$
  \[
      P(B|C)=P(A\cap B|C)+P(B\setminus A|C)
  \]
  Since their intersection is $A$ due to it being the subset
  \[
      P(B|C)=P(A|C)+P(B\setminus A|C)
  \]
  By the first axiom
  \[
      P(B\setminus A|C)\geq 0
  \]
  As such
  \[
      P(B|C)\geq P(A|C)
  \]
\end{proof}
This introduces some special cases
\begin{align*}
    P(A|B)&=\frac{P(\emptyset)}{P(B)}=0 \text{, for } A\cap B=\emptyset \\
    P(A|B)&=\frac{P(B)}{P(B)}=1 \text{, for } B\subset A \\
    P(A|B)&=\frac{P(A)}{P(B)} \text{, for } A\subset B
\end{align*}
Furthermore we also write the chain rule for conditional probability using the definition as a starting point
\begin{theorem}
  The extended chain rule is given by $P(A_{1}\cap A_{2}\cap A_{3}\cap\ldots\cap A_{n})=P(A_{1})P(A_{2}|A_{1})P(A_{3}|A_{2},A_{1})\cdots P(A_{n}|A_{n-1},A_{n-2},\ldots,A_{1})$
\end{theorem}
\begin{proof}
  From the definition of conditional probability we have that
  \[
      P(A|B)=\frac{P(A\cap B)}{P(B)}
  \]
  By isolation for $P(A\cap B)$ we get
  \[
      P(A\cap B)=P(A|B)P(B)=P(B|A)P(B)
  \]
  Extending to 3 or more events we get that
  \[
      P(A\cap B\cap C)=P(A\cap(B\cap C))=P(A)P(B\cap C|A)
  \]
  Applying the first equation
  \[
      P(B\cap C)=P(B)P(C|B)
  \]
  By conditioning both sides on $A$ we get
  \[
      P(B\cap C|A)=P(B|A)P(C|A,B)
  \]
  Inserting in the original equation we then get
  \[
      P(A\cap B\cap C)=P(A)P(B|A)P(C|A,B)
  \]
  Which can be generalised to
  \begin{equation*}
    \begin{gathered}
      P(A_{1}\cap A_{2}\cap\ldots\cap A_{n})=P(A_{1})P(A_{2}|A_{1})P(A_{3}|A_{2},A_{1})\cdots \\
      P(A_{n}|A_{n-1},A_{n-2},\ldots,A_{1})
    \end{gathered}
  \end{equation*}
\end{proof}
\subsubsection{Independence}
Conditional probabilities are only relevant if two events are not independent.
\begin{definition}
  Two events $A,B$ are independent if $P(A\cap B)=P(A)P(B)$.
\end{definition}
Independence for two or more events requires that all the individual events are independents, as well as all of them together, this means that for 3 events, all of the following must hold
\begin{align*}
    P(A\cap B)&=P(A)P(B) \\
    P(A\cap C)&=P(A)P(C) \\
    P(B\cap C)&=P(B)P(C) \\
    P(A\cap B\cap C)&=P(A)P(B)P(C)
\end{align*}
\begin{theorem}
    If $A$ and $B$ are independent, then $A$ and $\overline{B}$, $\overline{A}$ and $B$, $\overline{A}$ and $\overline{B}$ are also independent.
\end{theorem}
\begin{proof}
  The first statement is proven as the others can be concluded from it. As the statement is equivalent to
  \[
      P(A\cap \overline{B})=P(A)-P(A\cap B)
  \]
  As we know $A$ and $B$ are independent we have
  \[
      P(A\cap\overline{B})=P(A)-P(A)P(B)
  \]
  We factor out $P(A)$ and get that
  \[
      P(A\cap\overline{B})=P(A)(1-P(B))
  \]
  And as
  \[
      1-P(B)=P(\overline{B})
  \]
  It becomes clear that
  \[
      P(A\cap\overline{B})=P(A)P(\overline{B})
  \]
\end{proof}
To determine the probability of several unions of independent events, we make use of De Morgans law
\[
    P\left(\bigcup_{i=1}^{n} A_{i}\right)=1-P\left(\bigcap_{i=1}^{n} \overline{A_{i}}\right)
\]
Which is equivalent to
\[
    P\left(\bigcup_{i=1}^{n} A_{i}\right)=1-\prod_{i=1}^{n} (1-P(A_{i}))
\]
\subsubsection{Law of Total Probability}
The law of total probability states that the probability of an event, $A$ must be the sum of the probability of it occuring in every partition, $B_{1},B_{2},\ldots,B_{n}$ of the sample space
\[
    P(A)=\sum_{i=1}^{n}P(A\cap B_{i})=\sum_{i=1}^{n}P(A|B_{i})P(B_{i})
\]
\begin{proof}
  As $B_{1},B_{2},\ldots,B_{n}$ are partitions of the sample space we write
  \[
      S=\bigcup_{i=1}^{n} B_{i}
  \]
  Now the event $A$ occuring must be given by its intersection of the sample space
  \[
      A=A\cap\left(\bigcup_{i=1}^{n} B_{i}\right)
  \]
  By the distributive property it becomes clear that
  \[
      A=\bigcup_{i=1}^{n} (A\cap B_{i})
  \]
  Now as the partitions by definition are disjoint we can determine the probability as the sum of probabilities
  \[
      P(A)=P\left(\bigcup_{i=1}^{n} (A\cap B_{i})\right)=\sum_{i=1}^{n}P(A\cap B_{i})
  \]
  Rewriting using the definition of conditional probability (as $A\in B_{i}$ can only occur if $B_{i}$ has occured) we get
  \[
      P(A)=\sum_{i=1}^{n}P(A|B_{i})P(B_{i})
  \]
\end{proof}
\begin{theorem}
  Bayes rule states that: $P(B|A)=\frac{P(A|B)P(B)}{P(A)}$
\end{theorem}
\begin{proof}
  From the definition of conditional probability
  \[
      P(A|B)=\frac{P(A\cap B)}{P(B)}
  \]
  Multiplying by $P(B)$ on both sides
  \[
      P(A|B)P(B)=P(A\cap B)
  \]
  Dividing by $P(A)$ on both sides
  \[
      \frac{P(A|B)P(B)}{P(A)}=\frac{P(A\cap B)}{P(A)}=P(B|A)
  \]
\end{proof}
\subsubsection{Conditional Independence}
\begin{definition}
  Two events $A$ and $B$ are conditionally independent given an event $C$ if $P(A\cap B|C)=P(A|C)P(B|C)$.
\end{definition}
\begin{proof}
  From the definition of conditional probability we have
  \[
      P(A|B)=\frac{P(A\cap B)}{P(B)}
  \]
  Conditioning both sides on $C$ it becomes apparent that
  \[
      P(A|B,C)=\frac{P(A\cap B|C)}{P(B|C)}
  \]
  Assuming that $A$ and $B$ are conditionally independent we have
  \[
      P(A|B,C)=\frac{P(A|C)P(B|C)}{P(B|C)}=P(A|C)
  \]
\end{proof}
\subsection{Chapter resume (TO WRITE)}
